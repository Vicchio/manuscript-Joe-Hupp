%%%%%%%%%%%%%%%%%%%%%%%%%%%%%%%%%%%%%%%%%%%%%%%%%%%%%%%%%%%%%%%%%%%%%
%% This is a (brief) model paper using the achemso class
%% The document class accepts keyval options, which should include
%% the target journal and optionally the manuscript type. 
%%%%%%%%%%%%%%%%%%%%%%%%%%%%%%%%%%%%%%%%%%%%%%%%%%%%%%%%%%%%%%%%%%%%%
\documentclass[journal=jacsat,manuscript=article]{achemso}
\SectionNumbersOn
%\usepackage[letterpaper,left=0.5in,right=0.5in,top=1.0in,bottom=1.0in]{geometry}

%%%%%%%%%%%%%%%%%%%%%%%%%%%%%%%%%%%%%%%%%%%%%%%%%%%%%%%%%%%%%%%%%%%%%
%% Place any additional packages needed here.  Only include packages
%% which are essential, to avoid problems later. Do NOT use any
%% packages which require e-TeX (for example etoolbox): the e-TeX
%% extensions are not currently available on the ACS conversion
%% servers. 
%%%%%%%%%%%%%%%%%%%%%%%%%%%%%%%%%%%%%%%%%%%%%%%%%%%%%%%%%%%%%%%%%%%%%
\usepackage[version=3]{mhchem} % Formula subscripts using \ce{}
\usepackage{siunitx} % generating degrees Celsius in the document 
\usepackage{color}
\usepackage{soul} % allows highlighting text 
\usepackage{makecell}
\usepackage{booktabs}
\usepackage{amsmath}
\usepackage{amssymb}
\usepackage{todonotes}
\usepackage{gensymb}
\usepackage{verbatim}
\usepackage{hyperref}
\hypersetup{
    colorlinks=true,
    citecolor= red,
    linkcolor=blue,
    urlcolor=blue, 
    breaklinks=true
}
\usepackage{ulem}
\usepackage{float}

% Packages from the old SI section 
\usepackage{breakcites}
%\usepackage{caption}
\usepackage{float}
\usepackage[hang]{subfigure}
\usepackage{overpic}
\usepackage{listings}
\usepackage[super]{nth}
\usepackage{multicol}
\linespread{1.15}

%%%%%%%%%%%%%%%%%%%%%%%%%%%%%%%%%%%%%%%%%%%%%%%%%%%%%%%%%%%%%%%%%%%%%
%% If issues arise when submitting your manuscript, you may want to
%% un-comment the next line.  This provides information on the
%% version of every file you have used.
%%%%%%%%%%%%%%%%%%%%%%%%%%%%%%%%%%%%%%%%%%%%%%%%%%%%%%%%%%%%%%%%%%%%%
%%\listfiles

%%%%%%%%%%%%%%%%%%%%%%%%%%%%%%%%%%%%%%%%%%%%%%%%%%%%%%%%%%%%%%%%%%%%%
%% Place any additional macros here.  Please use \newcommand* where
%% possible, and avoid layout-changing macros (which are not used
%% when typesetting).
%%%%%%%%%%%%%%%%%%%%%%%%%%%%%%%%%%%%%%%%%%%%%%%%%%%%%%%%%%%%%%%%%%%%%
\newcommand*\mycommand[1]{\texttt{\emph{#1}}}
\DeclareRobustCommand
  \Compactcdots{\mathinner{\cdotp\mkern-2mu\cdotp\mkern-2mu\cdotp}}

%%%%%%%%%%%%%%%%%%%%%%%%%%%%%%%%%%%%%%%%%%%%%%%%%%%%%%%%%%%%%%%%%%%%%
%% Meta-data block
%% ---------------
%% Each author should be given as a separate \author command.
%%
%% Corresponding authors should have an e-mail given after the author
%% name as an \email command. Phone and fax numbers can be given
%% using \phone and \fax, respectively; this information is optional.
%%
%% The affiliation of authors is given after the authors; each
%% \affiliation command applies to all preceding authors not already
%% assigned an affiliation.
%%
%% The affiliation takes an option argument for the short name.  This
%% will typically be something like "University of Somewhere".
%%
%% The \altaffiliation macro should be used for new address, etc.
%% On the other hand, \alsoaffiliation is used on a per author basis
%% when authors are associated with multiple institutions.
%%%%%%%%%%%%%%%%%%%%%%%%%%%%%%%%%%%%%%%%%%%%%%%%%%%%%%%%%%%%%%%%%%%%%
\author{Stephen P. Vicchio}
\affiliation[Clemson University]
{Department of Chemical and Biomolecular Engineering, Clemson University, Clemson, SC}
\author{Zhihengyu Chen}
\affiliation[Stony Brook University]
{Department of Chemistry, Stony Brook University, Stony Brook, NY}
\author{Karena Chapman}
\email{karena.chapman@stonybrook.edu}
\affiliation[Stony Brook University]
{Department of Chemistry, Stony Brook University, Stony Brook, NY}
\author{Rachel B. Getman}
\email{rgetman@clemson.edu}
\affiliation[Clemson University]
{Department of Chemical and Biomolecular Engineering, Clemson University, Clemson, SC}

%%%%%%%%%%%%%%%%%%%%%%%%%%%%%%%%%%%%%%%%%%%%%%%%%%%%%%%%%%%%%%%%%%%%%
%% The document title should be given as usual. Some journals require
%% a running title from the author: this should be supplied as an
%% optional argument to \title.
%%%%%%%%%%%%%%%%%%%%%%%%%%%%%%%%%%%%%%%%%%%%%%%%%%%%%%%%%%%%%%%%%%%%%
\title[manuscript]{
Modeling the ligand environment of a supported four ion \ce{Ni} cluster under different operating conditions
}

%%%%%%%%%%%%%%%%%%%%%%%%%%%%%%%%%%%%%%%%%%%%%%%%%%%%%%%%%%%%%%%%%%%%%
%% Some journals require a list of abbreviations or keywords to be
%% supplied. These should be set up here, and will be printed after
%% the title and author information, if needed.
%%%%%%%%%%%%%%%%%%%%%%%%%%%%%%%%%%%%%%%%%%%%%%%%%%%%%%%%%%%%%%%%%%%%%
\abbreviations{IR,NMR,UV}
\keywords{American Chemical Society, \LaTeX}

%%%%%%%%%%%%%%%%%%%%%%%%%%%%%%%%%%%%%%%%%%%%%%%%%%%%%%%%%%%%%%%%%%%%%
%% The manuscript does not need to include \maketitle, which is
%% executed automatically.
%%%%%%%%%%%%%%%%%%%%%%%%%%%%%%%%%%%%%%%%%%%%%%%%%%%%%%%%%%%%%%%%%%%%%
\begin{document}

%%%%%%%%%%%%%%%%%%%%%%%%%%%%%%%%%%%%%%%%%%%%%%%%%%%%%%%%%%%%%%%%%%%%%
%% The "tocentry" environment can be used to create an entry for the
%% graphical table of contents. It is given here as some journals
%% require that it is printed as part of the abstract page. It will
%% be automatically moved as appropriate.
%%%%%%%%%%%%%%%%%%%%%%%%%%%%%%%%%%%%%%%%%%%%%%%%%%%%%%%%%%%%%%%%%%%%%
%\begin{tocentry}
%
%Some journals require a graphical entry for the Table of Contents.
%This should be laid out ``print ready'' so that the sizing of the
%text is correct.
%
%Inside the \texttt{tocentry} environment, the font used is %Helvetica
%8\,pt, as required by \emph{Journal of the American Chemical
%Society}.
%
%The surrounding frame is 9\,cm by 3.5\,cm, which is the maximum
%permitted for  \emph{Journal of the American Chemical Society}
%graphical table of content entries. The box will not resize if the
%content is too big: instead it will overflow the edge of the box.
%
%This box and the associated title will always be printed on a
%separate page at the end of the document.
%
%\end{tocentry}

%%%%%%%%%%%%%%%%%%%%%%%%%%%%%%%%%%%%%%%%%%%%%%%%%%%%%%%%%%%%%%%%%%%%%
%% The abstract environment will automatically gobble the contents
%% if an abstract is not used by the target journal.
%%%%%%%%%%%%%%%%%%%%%%%%%%%%%%%%%%%%%%%%%%%%%%%%%%%%%%%%%%%%%%%%%%%%%

%%%%%%%%%%%%%%%%%%%%%%%%%%%%%%%%%%%%%%%%%%%%%%%%%%%%%%%%%%%%%%%%%%%%%
%% Start the main part of the manuscript here.
%%%%%%%%%%%%%%%%%%%%%%%%%%%%%%%%%%%%%%%%%%%%%%%%%%%%%%%%%%%%%%%%%%%%%

%%%%%%%%%%%%%%%%%%%%%%%%%%%%%%%%%%%%%%%%%%%%%%%%%%%%%%%%%%%%%%%%%%%%%
%% Start the table of content here
%%%%%%%%%%%%%%%%%%%%%%%%%%%%%%%%%%%%%%%%%%%%%%%%%%%%%%%%%%%%%%%%%%%%%
\newpage
\tableofcontents


%%%%%%%%%%%%%%%%%%%%%%%%%%%%%%%%%%%%%%%%%%%%%%%%%%%%%%%%%%%%%%%%%%%%%
%% Computational Methodology
%%%%%%%%%%%%%%%%%%%%%%%%%%%%%%%%%%%%%%%%%%%%%%%%%%%%%%%%%%%%%%%%%%%%%
\newpage
\section{Computational Methodology}
\subsection{Schematic for \textit{ab initio} thermodynamic analysis}
    \begin{figure}[H]
        \centering
        \includegraphics[width=0.80\textwidth]{zi-images/00-General-Graphics/FPT-schematic-full.png}
        \caption{Diagram showing the assumed equilibria necessary to transform with \ce{H}, \ce{OH}, \ce{Ni} for the \textit{ab initio} thermodynamic analysis method. The assume assumed equilibria capture the operating conditions influences on the stability of the \ce{Ni} ion cluster.}
        \label{fig:SI-FPT-process-diagram}
    \end{figure}

    Figure~\ref{fig:SI-FPT-process-diagram} shows a schematic for \textit{ab initio} thermodynamic analysis, which is used to evaluate the stability of a \ce{Ni4}-cluster supported on NU-1000. We assume the \ce{Ni} ion cluster is in equilibrium with three different reservoirs, which allows for transformations with respect to \ce{H}, \ce{O}, and \ce{Ni}. Equilibrium with the bulk \ce{Ni} reservoir is turned off by assuming $\Delta N_{\ce{Ni}} = 0$, implying that the number of \ce{Ni} ions is fixed at four. We derive expressions relating the chemical potentials for each transformed species to their respective bulk reservoirs.   

%%%%%%%%%%%%%%%%%%%%%%%%%%%%%%%%%%%%%%%%%%%%%%%%%%%%%%%%%%%%%%%%%%%%%
%% Free Energy Expression
%%%%%%%%%%%%%%%%%%%%%%%%%%%%%%%%%%%%%%%%%%%%%%%%%%%%%%%%%%%%%%%%%%%%%
\newpage
\subsection{Deriving the free energy expressions}
We transform the free energy from a fixed number of atoms to a fixed chemical potential to account for the compositional variation of different activated clusters (Eq.~\ref{eq:transformation1}):
\begin{equation}
    F^{(0)}(V,T,N_{\text{H}},N_{\text{O}},N_{\text{Ni}}) \rightarrow 
    F^{(2)}(V,T,\mu_{\text{H}}, \mu_{\text{O}}, N_{\text{Ni}})    \rightarrow F^{(3)}(V,T,\mu_{\text{H}},\mu_{\text{O}},\mu_{\text{Ni}})
    \label{eq:transformation1}
\end{equation}
where $F$ is the Helmholtz free energy, $V$ is volume, $T$ is temperature, $\mu$ is chemical potential, and $N$ is the number of \ce{O}, \ce{H}, and \ce{Ni} atoms. The (2) superscript on $F^{(2)}$ indicates the second Legendre transform of $F^{(0)}$,\cite{Alberty1997} i.e., of $N_{\ce{H}}$ and $N_{\ce{O}}$ to $\mu_{\ce{H}}$ and $\mu_{\ce{O}}$, respectively. Additionally, the (3) superscript on $F^{(3)}$ indicates the Legendre transform of $F^{(2)}$ from $N_{\ce{Ni}}$ to $\mu_{\ce{Ni}}$. We write the transformed free energy expression for $F^{(2)}$ as Eq.~\ref{eq:free-energy-ind} and $F^{(3)}$ as Eq.~\ref{eq:free-energy-ind-Ni}. 
\begin{equation}
    \begin{split}
        F^{(2)}(V,T,\mu_{\text{H}},\mu_{\text{O}},N_{\text{Ni}}) =  F^{(0)}(V,T,N_{\text{H}},N_{\text{O}},N_{\text{Ni}}) - (\mu_{\text{H}})(N_{\text{H}}) - (\mu_{\text{O}})(N_{\text{O}})  \\ 
    \end{split}
    \label{eq:free-energy-ind}
\end{equation}
\begin{equation}
    \begin{split}
        F^{(3)}(V,T,\mu_{\text{H}},\mu_{\text{O}},\mu_{\text{Ni}}) = F^{(2)}(V,T,\mu_{\text{H}},\mu_{\text{O}},N_{\text{Ni}}) - (\mu_{\text{Ni}})(N_{\text{Ni}}) \\ 
    \end{split}
    \label{eq:free-energy-ind-Ni}
\end{equation}


%The difference in free energy between structures is then calculated by taking the difference between two structures, denoted modified and reference. 

%The free energy expressions in Eq.~\ref{eq:free-energy-ind} and $F^{(3)}$ as Eq.~\ref{eq:free-energy-ind-Ni} 

%We write the transformed free energy expression for $F^{(3)}$ as Eq.~\ref{eq:free-energy-trans-Ni}:

%where \hl{LABEL ALL OF THE DIFFERENT VARIABLES HERE}

The final transformed free energy expression ($\Delta F^{(2)}$) is computed relative to the reference structure (\hl{as shown in the main text in} Figure~\ref{fig:Ni-MOF-structures}a). We write the transformed free energy expression for $\Delta F^{(2)}$ as Eq.~\ref{eq:free-energy-trans}: 
\begin{equation}
    \begin{split}
        \Delta F^{(2)}(V,T,\mu_{\text{H}},\mu_{\text{O}},N_{\text{Ni}})  
        & = \Delta F^{(0)}(V,T,N_{\text{H}},N_{\text{O}},N_{\text{Ni}}) - (\mu_{\text{H}})(\Delta N_{\text{H}}) 
        - (\mu_{\text{O}})(\Delta N_{\text{O}})  \\ 
    \end{split}
    \label{eq:free-energy-trans}
\end{equation}
where $\Delta N_{\text{H}}$ and $\Delta N_{\text{O}}$ are computed by taking the difference in the number of \ce{O} and \ce{H} atoms of the cluster relative to the reference structure. A similar expression is constructed for $F^{(3)}$ (Eq.~\ref{eq:free-energy-trans-Ni}):
\begin{equation}
    \begin{split}
        \Delta F^{(3)}(V,T,\mu_{\text{H}},\mu_{\text{O}},\mu_{\text{Ni}})  
        = \Delta F^{(2)}(V,T,\mu_{\text{H}},\mu_{\text{O}},N_{\text{Ni}}) - (\mu_{\text{Ni}})(\Delta N_{\text{Ni}}) \\ 
    \end{split}
    \label{eq:free-energy-trans-Ni}
\end{equation}
where $\Delta N_{\text{Ni}}$ is also computed by taking the difference in the number of \ce{Ni} atoms of the cluster relative to the reference structure. 

%%%%%%%%%%%%%%%%%%%%%%%%%%%%%%%%%%%%%%%%%%%%%%%%%%%%%%%%%%%%%%%%%%%%%
%% Equilibrium Expressions
%%%%%%%%%%%%%%%%%%%%%%%%%%%%%%%%%%%%%%%%%%%%%%%%%%%%%%%%%%%%%%%%%%%%%
\newpage
\subsection{Equilibrium expressions for chemical potential terms}
For \ce{H}, the assumed equilibrium is with a reservoir of \ce{H2} gas:
\begin{equation}
    \frac{1}{2} H_{2}(g) \ce{<=>} H
\end{equation}
where the chemical potential terms are related by: 
\begin{equation}
    \frac{1}{2} \mu_{\ce{H2}}^{g}(T,P) = \mu_{\ce{H}}
\end{equation}  
\hl{The $\mu_{\ce{H2}}^{g}(T,P)$ is computed by  correcting the electronic energy (referenced at $T$=0 K) of an isolated molecule with the gas-phase Gibbs free energy values at a specific temperature and pressure} (shown in Eq.~S\ref{H2-to-reaction-conditions}). 
\begin{equation}
    \begin{split}
         \mu_{\ce{H2}}^{g}(T,P) &= E_{\ce{H2}}^{DFT} + E_{\ce{H2}}^{ZPE}+ \Delta \mu_{\ce{H2}}(T,P)  \\
    \end{split}
    \label{H2-to-reaction-conditions}
\end{equation}
In our analysis, $\mu_{\ce{H2}}(T,P)$ is an independent variable; we varied $\mu_{\ce{H2}}(T,P)$ to generate the \hl{phase diagrams shown in} Figure~\ref{fig:Ni-structure-diagram}. Furthermore, the $\mu_{\ce{H2}}^{g}(T,P)$ can be related to the temperature and pressure conditions of the system by relating the term to the free energy of the gas phase species using Eq.~S\ref{mu_2_G}:
\begin{equation}
    \begin{split}
         \Delta \mu_{\ce{H2}}(T,P) &= \Delta G_{H_{2}}(T,P)\\
         \Delta \mu_{\ce{H2}}(T,P) &= \Big[ \Delta G_{\ce{H2}}(T,P^{o})  + RT \ln{ \frac{P_{\ce{H2}}}{P_{\ce{H2}}^{o}}} \Big]  \\ 
    \end{split}
    \label{mu_2_G}
\end{equation}
where $\Delta G_{\ce{H2}}(T,P^{o})$ is the free energy referenced to 0 K, $R$ is gas constant, $T$ is the temperature, $P_{\ce{H2}}$ is the partial pressure of \ce{H2} gas and $P_{\ce{H2}}^{o}$ is the standard pressure (1 bar).  

\hl{With all calculations being performed at 0 K, the $G_{H_{2}}(T,P)$ was referenced to 0 K when evaluating the free energy from pMuTT. The electronic energy ($E_{H_{2}}^{DFT}$) were calculated in CP2K using the same computational parameters as described in the methodology section of the manuscript.} 



\newpage
\subsection{Deriving gas phase free energies values from empirical data}
The NASA Polynomials\hl{REF-XXX} are defined for a specific temperature range \hl{(i.e., from XXX to XXX)}. To correct the electronic energies from DFT, the gas phase species must be referenced to approximately 0 K. The standard reference state for the NASA Polynomials is 298 K; therefore, the empirical methods are approximated to 0 K using Eq.~S\ref{eq:pmutt-expression} by referencing the values at 10 K, following prior work.\hl{REF-Getman-FPT-Paper}
\begin{equation}
    \begin{split}
        \Delta G(T, P^{o}) &= [G(T,P^{o}) - G(T=298 K, P^{o})] - [G(T=10 K, P^{o}) - G(T=298 K, P^{o})] \\
        \Delta G(T, P^{o}) &= [G(T,P^{o})] - [G(T=10 K, P^{o})] 
    \end{split}
    \label{eq:pmutt-expression}
\end{equation} 
where $\Delta G(T, P^{o})$ is the free energy referenced at $T = 10$~K, and $G(T,P^{o})$ is the free energy referenced at $T = 298$~K. Figures~\ref{fig:h2-pmutt-expression} and \ref{fig:h2o-pmutt-expression} show the extrapolation of the NASA Polynomial referenced to 0~K. 

\begin{figure}[H]
    \centering
    \includegraphics[width=0.70\textwidth]{zi-images/00-General-Graphics/2021-figure-H2-pMuTT.png}
    \caption{The gas phase free energy of \ce{H2} as a function of temperature. The reference state it defined as 10~K. The red line shows the NASA Polynomial within defined range, and the blue line shows the extrapolation.}
    \label{fig:h2-pmutt-expression}
\end{figure}

\begin{figure}[H]
    \centering
    \includegraphics[width=0.70\textwidth]{zi-images/00-General-Graphics/2021-figure-H2O-pMuTT.png}
    \caption{The gas phase free energy of \ce{H2O} as a function of temperature. The reference state it defined as 10~K. The red line shows the NASA Polynomial within defined range, and the blue line shows the extrapolation.}    \label{fig:h2o-pmutt-expression}
\end{figure}


\newpage
\subsection{Identifying spin contamination in DFT calculations}
The library was optimized at different \ce{Ni} spin states; the considered spin states for \ce{Ni} ions were singlet (no unpaired electrons) and triplet (two unpaired electrons). With the cluster initially containing four \ce{Ni} ions, the intermediate combinations were calculated (one singlet and three triplet, two singlet and two triplet, etc.). Structures that exhibited spin contamination were removed from the analysis. Comparisons between the ideal and single determinant $S^{**}2$ values were inspected. If the single determinant value deviated by more than 10\% of the ideal determinant the structure was removed from analysis. 
\begin{center}
\begin{table}[H]
\centering
  \setlength\tabcolsep{8pt}
  \caption{Example of Spin Contamination Analysis for \ce{Ni4(OH)6 \cdot 2H2O} Structure.}
  \label{tbl:spin_contamination}
  \begin{tabular}{lcccc}
    \hline
        \ce{Ni(II)} Spin Configuration  & \thead{ Electronic \\ Energy (hartree)} &  Ideal $S^{**}2$ &   Single $S^{**}2$ & Contamination? \\
        \hline
        a) Four singlet (RKS\textsuperscript{a}) & -4778.84071 & N\/a & N\/a & No \\
        b) Four singlet (UKS)                    & -4778.87973 & 0.000  & 2.386  & Yes \\
        c) Three singlet, one triplet            & -4778.88313 & 2.000  & 3.978  & Yes \\
        d) Two singlet, two triplet              & -4778.88042 & 6.000  & 6.457  & Yes \\
        e) One singlet, three triplet            & -4778.88209 & 12.000 & 12.014 & No \\
        f) Four triplet                          & -4778.84120 & 20.000 & 20.013 & No \\
        \hline
    \end{tabular} \\
    \textsuperscript{a} Restricted Kohn-Sham \\
\end{table}    
\end{center}
From Table \ref{tbl:spin_contamination}, the lowest energy structure is c) three singlet, one triplet \ce{Ni} ions; however, the structure is removed from the analysis because the ideal $S^{**}2$ and single $S^{**}2$ deviate. Therefore, the lowest energy structure for this particular configuration is the e) one singlet, three triplet \ce{Ni} ions. 

\newpage
\subsection{Frequency Calculations}
We fix the organic linker and inorganic nodes, thereby only calculating only the \ce{Ni} cluster vibrational contributions to the free energy ($F^\text{vib}$). We ensure that the same atoms are fixed across all frequency calculations in order to correctly compare ($F^\text{vib}$). When constructing the vibrational partition function, all frequencies less than 50 cm\textsuperscript{-1} are replaced with 50 cm\textsuperscript{-1} to correct for the breakdown in the harmonic oscillator approximation for low frequency vibrational modes.\cite{Ribeiro2011} The partitions functions are computed in the pMuTT\cite{LYM2019106864} Python package to compute $E^\text{ZP}$ and $F^\text{vib}$ for all structures. A sample input file for the CP2K frequency calculations is provided in the Supporting Information.

\newpage
\subsection{\ce{Ni-O} Coordination Numbers}
We report the \ce{Ni-O} coordination numbers for structures appearing on the phase diagram. The \ce{Ni-O} coordination numbers are determined by geometric criteria defined by the \ce{Ni-O} distance, which is between 1.9 {\AA} and 2.4 {\AA} for \ce{OH} and\ce{O} ligands. For \ce{H2O} ligands, we add an additional geometric criteria that includes the orientation of the \ce{H2O} ligand relative to the cluster. If the \ce{Ni-O} distance is below 2.4 {\AA}, the \ce{H2O} is coordinated. Additionally for \ce{H2O} ligands, if the \ce{Ni-O} distance is below 3.3 and the \ce{Ni-O} distance is less than the \ce{Ni-H} distances is coordinated to the cluster. We report the coordination numbers on a per \ce{Ni} basis, which is the average of the four \ce{Ni} ions present within the cluster.

\newpage
\subsection{Sample Input File for \ce{Ni4} Clusters - Optimization}
Below is a sample CP2K input file. If the singlet spin state was desired for all \ce{Ni(II)} atoms, UKS F and MULTIPLICITY 1 were used. Otherwise, UKS T was used with the appropriate MULTIPLICITY specified. The following sample input file is for optimizations. 
\begin{center}
    \lstset{numbers=left, basicstyle=\ttfamily, numbersep=-45pt, }
    \begin{lstlisting}[language=bash]
        &GLOBAL
           PRINT_LEVEL  MEDIUM
           PROJECT_NAME info-ni4oh6-4h2o-config2-MULTI9
           RUN_TYPE  GEO_OPT
           WALLTIME 71:47:00
         &END GLOBAL
         &MOTION
           &GEO_OPT
             TYPE  MINIMIZATION
             OPTIMIZER  BFGS
             MAX_ITER  2000
             MAX_DR     5.0000000000000001E-04
             MAX_FORCE     5.0000000000000002E-05
             RMS_DR     5.0000000000000001E-04
             RMS_FORCE     5.0000000000000002E-05
             STEP_START_VAL  0
             &BFGS
               TRUST_RADIUS     2.5000000000000006E-01
             &END BFGS
           &END GEO_OPT
         &END MOTION
         &FORCE_EVAL
           METHOD  QS
           STRESS_TENSOR  ANALYTICAL
           &DFT
             BASIS_SET_FILE_NAME BASIS_file
             POTENTIAL_FILE_NAME POTENTIALS_file
             UKS  T
             MULTIPLICITY  9
             CHARGE  0
             &SCF
               MAX_SCF  1000
               EPS_SCF     9.9999999999999995E-07
               SCF_GUESS  ATOMIC
               &OT  T
                 MINIMIZER  CG
                 PRECONDITIONER  FULL_ALL
                 ENERGY_GAP     1.0000000000000000E-03
               &END OT
               &OUTER_SCF  T
                 EPS_SCF     9.9999999999999995E-07
                 MAX_SCF  50
               &END OUTER_SCF
             &END SCF
             &QS
               EPS_DEFAULT     1.0000000000000000E-10
               METHOD  GPW
             &END QS
             &MGRID
               NGRIDS  5
               CUTOFF     3.6000000000000000E+02
               REL_CUTOFF     8.0000000000000000E+01
             &END MGRID
             &XC
               DENSITY_CUTOFF     1.0000000000000000E-10
               GRADIENT_CUTOFF     1.0000000000000000E-10
               TAU_CUTOFF     1.0000000000000000E-10
               &XC_FUNCTIONAL  NO_SHORTCUT
                 &PBE  T
                 &END PBE
               &END XC_FUNCTIONAL
               &VDW_POTENTIAL
                 POTENTIAL_TYPE  PAIR_POTENTIAL
                 &PAIR_POTENTIAL
                   TYPE  DFTD3(BJ)
                   PARAMETER_FILE_NAME dftd3.dat
                   REFERENCE_FUNCTIONAL PBE
                   CALCULATE_C9_TERM  F
                 &END PAIR_POTENTIAL
               &END VDW_POTENTIAL
             &END XC
           &END DFT
           &SUBSYS
             &CELL
               A     4.061108E+01   0.00000E+00    0.00000E+00
               B     2.03054E+01    3.51702E+01    0.00000E+00
               C     0.00000E+00    0.00000E+00    1.59897E+01
               MULTIPLE_UNIT_CELL  1 1 1
               SYMMETRY  MONOCLINIC_GAMMA_AB
             &END CELL
             &KIND C
               BASIS_SET DZVP-MOLOPT-SR-GTH-q4
               POTENTIAL GTH-PBE-q4
             &END KIND
             &KIND H
               BASIS_SET DZVP-MOLOPT-SR-GTH-q1
               POTENTIAL GTH-PBE-q1
             &END KIND
             &KIND Ni
               BASIS_SET DZVP-MOLOPT-SR-GTH-q18
               POTENTIAL GTH-PBE-q18
             &END KIND
             &KIND O
               BASIS_SET DZVP-MOLOPT-SR-GTH-q6
               POTENTIAL GTH-PBE-q6
             &END KIND
             &KIND Zr
               BASIS_SET DZVP-MOLOPT-SR-GTH-q12
               POTENTIAL GTH-PBE-q12
             &END KIND
             &TOPOLOGY
               COORD_FILE_NAME ./sNi4OH6-4H2O-config2.xyz
               COORD_FILE_FORMAT  XYZ
               NUMBER_OF_ATOMS  584
               CONN_FILE_FORMAT  OFF
               MULTIPLE_UNIT_CELL  1 1 1
             &END TOPOLOGY
           &END SUBSYS
           &PRINT
             &FORCES  ON
             &END FORCES
           &END PRINT
         &END FORCE_EVAL
    \end{lstlisting}
\end{center}
\newpage
\subsection{Sample Input File for \ce{Ni4} Clusters - Frequency}
Below is a sample CP2K input file for frequency calculations. The $&$MOTION section is used to freeze all framework and node atoms in the MOF, thereby calculating only the vibrational modes of the cluster atoms. The xyz file is formatted so that the frozen atoms are the first 548 atoms of the file. 

\begin{center}
    \lstset{numbers=left, basicstyle=\ttfamily, numbersep=-45pt, }
    \begin{lstlisting}[language=bash]
     &GLOBAL
       PRINT_LEVEL  MEDIUM
       PROJECT_NAME info-000-014_Zr18_C264_Ni4_O96_H178_UKS-multi5-freq
       RUN_TYPE VIBRATIONAL_ANALYSIS
       WALLTIME 71:47:00
     &END GLOBAL
     &VIBRATIONAL_ANALYSIS
       FULLY_PERIODIC T
       NPROC_REP 16
     &END VIBRATIONAL_ANALYSIS
     &MOTION
        &CONSTRAINT
          &FIXED_ATOMS
              COMPONENTS_TO_FIX  XYZ
              LIST  1..548
          &END FIXED_ATOMS
       &END CONSTRAINT
     &END MOTION
     &FORCE_EVAL
       METHOD  QS
       STRESS_TENSOR  ANALYTICAL
       &DFT
         BASIS_SET_FILE_NAME BASIS_file
         POTENTIAL_FILE_NAME POTENTIALS_file
         UKS T
         MULTIPLICITY 5
         CHARGE  0
         &SCF
           MAX_SCF  1000
           EPS_SCF     9.9999999999999995E-07
           SCF_GUESS  ATOMIC
           &OT  T
             MINIMIZER  CG
             PRECONDITIONER  FULL_ALL
             ENERGY_GAP     1.0000000000000000E-03
           &END OT
           &OUTER_SCF  T
             EPS_SCF     9.9999999999999995E-07
             MAX_SCF  50
           &END OUTER_SCF
         &END SCF
         &QS
           EPS_DEFAULT     1.0000000000000000E-10
           METHOD  GPW
         &END QS
         &MGRID
           NGRIDS  5
           CUTOFF     3.6000000000000000E+02
           REL_CUTOFF     8.0000000000000000E+01
         &END MGRID
         &XC
           DENSITY_CUTOFF     1.0000000000000000E-10
           GRADIENT_CUTOFF     1.0000000000000000E-10
           TAU_CUTOFF     1.0000000000000000E-10
           &XC_FUNCTIONAL  NO_SHORTCUT
             &PBE  T
             &END PBE
           &END XC_FUNCTIONAL
           &VDW_POTENTIAL
             POTENTIAL_TYPE  PAIR_POTENTIAL
             &PAIR_POTENTIAL
               TYPE  DFTD3(BJ)
               PARAMETER_FILE_NAME dftd3.dat
               REFERENCE_FUNCTIONAL PBE
               CALCULATE_C9_TERM  F
             &END PAIR_POTENTIAL
           &END VDW_POTENTIAL
         &END XC
       &END DFT
       &SUBSYS
         &CELL
           A     4.0611054456722748E+01    0.0000000000000000E+00    0.0000000000000000E+00
           B     2.0305492374827001E+01    3.5170224956669564E+01    0.0000000000000000E+00
           C     0.0000000000000000E+00    0.0000000000000000E+00    1.5989790987837972E+01
           MULTIPLE_UNIT_CELL  1 1 1
           SYMMETRY  MONOCLINIC_GAMMA_AB
         &END CELL
         &KIND C
           BASIS_SET DZVP-MOLOPT-SR-GTH-q4
           POTENTIAL GTH-PBE-q4
         &END KIND
         &KIND H
           BASIS_SET DZVP-MOLOPT-SR-GTH-q1
           POTENTIAL GTH-PBE-q1
         &END KIND
         &KIND Ni
           BASIS_SET DZVP-MOLOPT-SR-GTH-q18
           POTENTIAL GTH-PBE-q18
         &END KIND
         &KIND O
           BASIS_SET DZVP-MOLOPT-SR-GTH-q6
           POTENTIAL GTH-PBE-q6
         &END KIND
         &KIND Zr
           BASIS_SET DZVP-MOLOPT-SR-GTH-q12
           POTENTIAL GTH-PBE-q12
         &END KIND
         &TOPOLOGY
           COORD_FILE_NAME 014_Zr18_C264_Ni4_O96_H178_UKS-multi5.xyz
           COORD_FILE_FORMAT  XYZ
           CONN_FILE_FORMAT  OFF
           MULTIPLE_UNIT_CELL  1 1 1
         &END TOPOLOGY
       &END SUBSYS
       &PRINT
         &FORCES  ON
         &END FORCES
       &END PRINT
     &END FORCE_EVAL
    \end{lstlisting}
\end{center}
\subsection{Sample Input File for bulk Ni}
Below is a sample CP2K input file. If the singlet spin state was desired for all \ce{Ni(II)} atoms, UKS F and MULTIPLICITY 1 were used. Otherwise, UKS T was used with the appropriate MULTIPLICITY specified.
\begin{center}
    \lstset{numbers=left, basicstyle=\ttfamily, numbersep=-45pt}
    \begin{lstlisting}[language=bash]
        &GLOBAL
          PRINT_LEVEL low
          PROJECT_NAME Ni-MULTI9
          RUN_TYPE cell_opt
        &END GLOBAL
        &MOTION
          &CELL_OPT
            EXTERNAL_PRESSURE [bar] 1.0
            MAX_DR 0.001
            MAX_FORCE 0.0001
            MAX_ITER 400
            OPTIMIZER BFGS
            PRESSURE_TOLERANCE [bar] 10.0
            RMS_DR 0.0003
            RMS_FORCE 0.00003
            TYPE direct_cell_opt
            &BFGS
              TRUST_RADIUS 0.1
              USE_MODEL_HESSIAN off
              USE_RAT_FUN_OPT on
            &END BFGS
          &END CELL_OPT
        &END MOTION
        &FORCE_EVAL
          METHOD QS
          STRESS_TENSOR analytical
          &DFT
            BASIS_SET_FILE_NAME ./BASIS_file
            POTENTIAL_FILE_NAME ./POTENTIALS_file
            &KPOINTS
              SCHEME MONKHORST-PACK 6 6 6
              FULL_GRID yes
              SYMMETRY yes
              VERBOSE yes
              PARALLEL_GROUP_SIZE -1
            &END KPOINTS
            &MGRID
              NGRIDS 5
              CUTOFF 400.0
              REL_CUTOFF 60.0
            &END MGRID
            &QS
              EPS_DEFAULT 1.0E-12
              EXTRAPOLATION use_prev_p
            &END QS
            &SCF
              ADDED_MOS 60
              EPS_SCF 1.0E-8
              MAX_SCF 300
              SCF_GUESS restart
              &DIAGONALIZATION yes
                ALGORITHM STANDARD
              &END DIAGONALIZATION
              &MIXING yes
                ALPHA 0.4
                BETA 1.0
                METHOD broyden_mixing
                NBROYDEN 8
              &END MIXING
              &SMEAR on
                METHOD FERMI_DIRAC
                ELECTRONIC_TEMPERATURE [K] 2000.0
              &END SMEAR
            &END SCF
            &XC
              &XC_FUNCTIONAL PBE
              &END XC_FUNCTIONAL
              &VDW_POTENTIAL
                POTENTIAL_TYPE pair_potential
                &PAIR_POTENTIAL
                  TYPE DFTD3(BJ)
                  PARAMETER_FILE_NAME dftd3.dat
                  REFERENCE_FUNCTIONAL PBE
                &END PAIR_POTENTIAL
              &END VDW_POTENTIAL
            &END XC
          &END DFT
          &SUBSYS
            &CELL
              ABC 3.50 3.50 3.50
              MULTIPLE_UNIT_CELL 2 2 2
            &END CELL
            &COORD
              SCALED
              Ni    0    0    0
              Ni    0  1/2  1/2
              Ni  1/2    0  1/2
              Ni  1/2  1/2    0
            &END COORD
            &KIND Ni
              BASIS_SET DZVP-MOLOPT-SR-GTH-q18
              POTENTIAL GTH-PBE-q18
            &END KIND
            &TOPOLOGY
              MULTIPLE_UNIT_CELL 2 2 2
            &END TOPOLOGY
          &END SUBSYS
        &END FORCE_EVAL
    \end{lstlisting}
\end{center}



%%%%%%%%%%%%%%%%%%%%%%%%%%%%%%%%%%%%%%%%%%%%%%%%%%%%%%%%%%%%%%%%%%%%%
%% ab initio thermodynamic analysis
%%%%%%%%%%%%%%%%%%%%%%%%%%%%%%%%%%%%%%%%%%%%%%%%%%%%%%%%%%%%%%%%%%%%%
\newpage
\section{\textit{ab initio} thermodynamic analysis}

%%%%%%%%%%%%%%%%%%%%%%%%%%%%%%%%%%%%%%%%%%%%%%%%%%%%%%%%%%%%%%%%%%%%%
%% Phase Diagram Structures
%%%%%%%%%%%%%%%%%%%%%%%%%%%%%%%%%%%%%%%%%%%%%%%%%%%%%%%%%%%%%%%%%%%%%
\newpage
\section{Phase Diagram Structures}
\subsubsection{Green}
    \begin{figure}[H]
        \centering
    %    \includegraphics[width=0.75\textwidth]{}
        \caption{The local structural information for the \ce{Ni4(OH)6.4H2O} (green) structure, presented as the (a) d-PDF, (b) the schematic representation, and (c) a 3D rendering. The key distances for \ce{Ni4(OH)6.4H2O} (green) based on d-PDF analysis include a \ce{Ni-O} peak at 2.02 {\AA}, a \ce{Ni{\Compactcdots}Ni} peak at 3.03 {\AA}, and a \ce{Ni{\Compactcdots}Zr} peak at 3.92 {\AA}, as seen in (a). The structure does not exhibit any \ce{Ni{\Compactcdots}Ni} or \ce{Ni{\Compactcdots}Zr} peak splitting.
        }
        \label{fig:SI-structure-diagram-blue}
    \end{figure}
    
\subsubsection{Purple}
    \begin{figure}[H]
        \centering
    %    \includegraphics[width=0.75\textwidth]{}
        \caption{The local structural information for the \ce{Ni4(OH)6.3H2O} (purple) structure, presented as the (a) d-PDF, (b) the schematic representation, and (c) a 3D rendering. The key distances for \ce{Ni4(OH)6.3H2O} (purple) based on d-PDF analysis include a \ce{Ni-O} peak at 2.07 {\AA}, a \ce{Ni{\Compactcdots}Ni} peak at 3.03 {\AA}, and a \ce{Ni{\Compactcdots}Zr} peak at 3.92 {\AA}, as seen in (a). The structure does not exhibit any \ce{Ni{\Compactcdots}Ni} or \ce{Ni{\Compactcdots}Zr} peak splitting. 
        }
        \label{fig:SI-structure-diagram-blue}
    \end{figure}  
    
\subsubsection{Blue}
    \begin{figure}[H]
        \centering
    %    \includegraphics[width=0.75\textwidth]{}
        \caption{The local structural information for the \ce{Ni4(OH)5(H).4H2O} (blue) structure, presented as the (a) d-PDF, (b) the schematic representation, and (c) a 3D rendering. The key distances for \ce{Ni4(OH)5(H).4H2O} (blue) based on d-PDF analysis include a \ce{Ni-O} peak at 1.93 {\AA}, \ce{Ni{\Compactcdots}Ni} peaks at 2.75, 3.05, and 3.37 {\AA}, and \ce{Ni{\Compactcdots}Zr} peaks at 3.87 {\AA}, as seen in (a). The structure exhibits a triplet of \ce{Ni{\Compactcdots}Ni} peaks and a broadening of the \ce{Ni-O} peak due to the chain of \ce{H2O} molecules. 
        }
        \label{fig:SI-structure-diagram-blue}
    \end{figure}

\subsubsection{Cyan}
    \begin{figure}[H]
        \centering
    %    \includegraphics[width=0.75\textwidth]{}
        \caption{The local structural information for the \ce{Ni4(OH)6.2H2O} (cyan) structure, presented as the (a) d-PDF, (b) the schematic representation, and (c) a 3D rendering. The key distances for \ce{Ni4(OH)6.2H2O} (cyan) based on d-PDF analysis include a \ce{Ni-O} peak at 2.03 {\AA}, \ce{Ni{\Compactcdots}Ni} peaks at 2.77 and 3.14 {\AA}, and a  \ce{Ni{\Compactcdots}Zr} peak at 3.88 {\AA}, as seen in (a). The structure exhibits split \ce{Ni{\Compactcdots}Ni} peaks that do not match the experimental d-PDF.
        }
        \label{fig:SI-structure-diagram-blue}
    \end{figure}
    
\subsubsection{Yellow}
    \begin{figure}[H]
        \centering
    %    \includegraphics[width=0.75\textwidth]{}
        \caption{The local structural information for the \ce{Ni4(OH)4(O)} (yellow) structure, presented as the (a) d-PDF, (b) the schematic representation, and (c) a 3D rendering. The key distances for \ce{Ni4(OH)4(O)} (yellow) based on d-PDF analysis include a \ce{Ni-O} peak at 2.00 {\AA}, \ce{Ni{\Compactcdots}Ni} peaks at 2.74 and 3.06 {\AA}, and \ce{Ni{\Compactcdots}Zr} peaks at 3.93 {\AA}, as seen in (a). The split \ce{Ni{\Compactcdots}Ni} peaks are not as pronounced as other structures, with the peaks being broad. 
        }
        \label{fig:SI-structure-diagram-blue}
    \end{figure}

\subsubsection{Orange}
    \begin{figure}[H]
        \centering
    %    \includegraphics[width=0.75\textwidth]{}
        \caption{The local structural information for the \ce{Ni4(OH)4.2H2O} (orange) structure, presented as the (a) d-PDF, (b) the schematic representation, and (c) a 3D rendering. The key distances for \ce{Ni4(OH)4.2H2O} (orange) based on d-PDF analysis include a \ce{Ni-O} peak at 2.02 {\AA}, \ce{Ni{\Compactcdots}Ni} peaks at 2.54 and 3.34 {\AA}, and a  \ce{Ni{\Compactcdots}Zr} peak at 3.98 {\AA}, as seen in (a). The peak at 2.54 {\AA} is from the middle \ce{Ni} ions having a low \ce{Ni-O} coordination number. 
        }
        \label{fig:SI-structure-diagram-blue}
    \end{figure}

\subsubsection{White}
    \begin{figure}[H]
        \centering
    %    \includegraphics[width=0.75\textwidth]{}
        \caption{The local structural information for the \hl{XXX} structure, presented as the (a) d-PDF, (b) the schematic representation, and (c) a 3D rendering. The key distances for \hl{XXX} based on d-PDF analysis include \ce{Ni-O} peaks at \hl{XXX} {\AA}, \ce{Ni{\Compactcdots}Ni} peaks at \hl{XXX} {\AA}, and \ce{Ni{\Compactcdots}Zr} peaks at \hl{XXX} {\AA}, as seen in (a). 
        }
        \label{fig:SI-structure-diagram-blue}
    \end{figure}

\subsubsection{Magenta}
    \begin{figure}[H]
        \centering
    %    \includegraphics[width=0.75\textwidth]{}
        \caption{The local structural information for the \hl{XXX} structure, presented as the (a) d-PDF, (b) the schematic representation, and (c) a 3D rendering. The key distances for \hl{XXX} based on d-PDF analysis include \ce{Ni-O} peaks at \hl{XXX} {\AA}, \ce{Ni{\Compactcdots}Ni} peaks at \hl{XXX} {\AA}, and \ce{Ni{\Compactcdots}Zr} peaks at \hl{XXX} {\AA}, as seen in (a). 
        }
        \label{fig:SI-structure-diagram-blue}
    \end{figure}

\subsubsection{Gray}
    \begin{figure}[H]
        \centering
    %    \includegraphics[width=0.75\textwidth]{}
        \caption{The local structural information for the \hl{XXX} structure, presented as the (a) d-PDF, (b) the schematic representation, and (c) a 3D rendering. The key distances for \hl{XXX} based on d-PDF analysis include \ce{Ni-O} peaks at \hl{XXX} {\AA}, \ce{Ni{\Compactcdots}Ni} peaks at \hl{XXX} {\AA}, and \ce{Ni{\Compactcdots}Zr} peaks at \hl{XXX} {\AA}, as seen in (a). 
        }
        \label{fig:SI-structure-diagram-blue}
    \end{figure}
    
\subsubsection{Red}
    \begin{figure}[H]
        \centering
    %    \includegraphics[width=0.75\textwidth]{}
        \caption{The local structural information for the \hl{XXX} structure, presented as the (a) d-PDF, (b) the schematic representation, and (c) a 3D rendering. The key distances for \hl{XXX} based on d-PDF analysis include \ce{Ni-O} peaks at \hl{XXX} {\AA}, \ce{Ni{\Compactcdots}Ni} peaks at \hl{XXX} {\AA}, and \ce{Ni{\Compactcdots}Zr} peaks at \hl{XXX} {\AA}, as seen in (a). 
        }
        \label{fig:SI-structure-diagram-blue}
    \end{figure}

%%%%%%%%%%%%%%%%%%%%%%%%%%%%%%%%%%%%%%%%%%%%%%%%%%%%%%%%%%%%%%%%%%%%%
%% Full d-PDF Diagrams for all structures on the phase diagram
%%%%%%%%%%%%%%%%%%%%%%%%%%%%%%%%%%%%%%%%%%%%%%%%%%%%%%%%%%%%%%%%%%%%%
\newpage
\section{d-PDFs for All Thermodynamic Minima}
% d-PDF diagram for all thermodynamic minima
\begin{figure}[H]
    \centering
%    \includegraphics[width=0.75\textwidth]{}
    \caption{d-PDFs for (a) the synthesized structure and (b) simulated structures representing thermodynamic minima with \ce{Ni-O} coordination numbers $\ge$ 4, and (c) simulated structures representing thermodynamic minima with \ce{Ni-O} coordination numbers $<$ 4. The d-PDFs presented in (c) are the structures that do not appear in the main text, given the observed d-PDFs. %The structure labels and colors for the simulated structures are the same as in the text and in Figures~\ref{fig:phase_diagram_Ni} and \ref{fig:Ni-structure-diagram}. 
    Gray dotted lines indicate key distances observed for the synthesized structure, which are also annotated on the figure.
    }
    \label{fig:SI-all-dPDFs}
\end{figure}

%%%%%%%%%%%%%%%%%%%%%%%%%%%%%%%%%%%%%%%%%%%%%%%%%%%%%%%%%%%%%%%%%%%%%
%% Weighted d-PDFs
%%%%%%%%%%%%%%%%%%%%%%%%%%%%%%%%%%%%%%%%%%%%%%%%%%%%%%%%%%%%%%%%%%%%%
\newpage
\section{Boltzmann Weighted d-PDFs}
We use Boltzmann weighting to combine the structural information of unique structures into a Boltzmann weighted d-PDF. We perform the Boltzmann weighting by computing the free energies of all the structures at specific \ce{H2} and \ce{H2O} chemical potentials, and then use Eq.~\ref{eq:weighted-dPDFs} to compute the Boltzmann factor ($P_{i}$) for each structure ${i}$.
\begin{equation}
    P_{i} = \frac{e^{\frac{-\Delta G_{i}}{k_{b}T}}}{\sum_{l} e^{\frac{-\Delta G_{l}}{k_{b}T}}}
    \label{eq:weighted-dPDFs}
\end{equation}
where $\Delta G$ is the free energy of structure computed according to Eq.~S\hl{XXX}. We assume a temperature of $T=298 \text{K}$. Note that the Boltzmann weighting considers all the d-PDFs in our library of structures; not just the structures that are thermodynamically relevant. For each structure, we then take each d-PDF between 0 {\AA} and 5 {\AA} multiplied by each Boltzmann factor and then sum all the d-PDFs to generate the weighted d-PDF at specific \ce{H2} and \ce{H2O} chemical potentials. Key weighted Boltzmann factors are shown in Figure~S\ref{fig:SI-all-dPDFs}.

% d-PDF diagram for all thermodynamic minima
\begin{figure}[H]
    \centering
   \includegraphics[width=0.50\textwidth]{zi-images/04-SI-images/2022-single-weighted_dPDF.png}
    \caption{The d-PDFs for (a) the synthesized structure and Botlzmann weighted d-PDFs (b-d). Gray dotted lines indicate key distances observed for the synthesized structure, which are also annotated on the figure.
    }
    \label{fig:SI-all-dPDFs}
\end{figure}

%%%%%%%%%%%%%%%%%%%%%%%%%%%%%%%%%%%%%%%%%%%%%%%%%%%%%%%%%%%%%%%%%%%%%
%% Structure Heat Maps
%%%%%%%%%%%%%%%%%%%%%%%%%%%%%%%%%%%%%%%%%%%%%%%%%%%%%%%%%%%%%%%%%%%%%
\newpage
\section{Phase Diagram Structure Heat Maps}
% Structure Diagram
\hl{Insert all of the heat map structures here for reference..just show the structures are they are for reference...}

%%%%%%%%%%%%%%%%%%%%%%%%%%%%%%%%%%%%%%%%%%%%%%%%%%%%%%%%%%%%%%%%%%%%%
%% The appropriate \bibliography command should be placed here.
%% Notice that the class file automatically sets \bibliographystyle
%% and also names the section correctly.
%%%%%%%%%%%%%%%%%%%%%%%%%%%%%%%%%%%%%%%%%%%%%%%%%%%%%%%%%%%%%%%%%%%%%
\newpage
\bibliography{achemso-demo}

\end{document}