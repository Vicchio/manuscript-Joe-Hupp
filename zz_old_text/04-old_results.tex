In contrast, the dominant structure in Figure~\ref{fig:phase_diagram_Ni_combined} (b) is structure 4 cite if previously identified; this structure features four Ni ions linked by hydroxyl ligands; however, the Ni in the center of the chain are undercoordinated due to loss of OH ligands via hydrogenation. In Figure 2b, lowering lowering of the temperature below 50\degrees C recovers the H2O ligands, first forming adsorbed H2O that turns into a chain of hydrogen bonded H2O molecules below 40\degrees C.

We explore the influence of $P_{\text{\ce{H2O}}}$ on the thermodynamic landscape of the \ce{Ni} cluster in the \hl{supporting information}. Overall, we demonstrate that a large $P_{\text{\ce{H2O}}}$ is required to obtains \ce{Ni} clusters with coordination numbers matches those observed experimentally. 

For example, going from structure 4 to 5 and 5 to 6 in Figure 2a... briefly describe what happens, e.g., the OH ligands that connect Ni ions to each other are hydrogenated; and it is more thermodynamically preferable for the resulting water ligands desorb and the coordinated Ni to agglomerate than for the H2O ligand to remain in the structure. When this happens, the Ni cations are destabilized and hence attempt to minimize their free energies via aggregation. As a result, the dominant structure in Figure~\ref{fig:phase_diagram_Ni_combined} (a), structure 6-\ce{Ni2(H)2} (lilac), comprises isolated nickel hydride species; interestingly, such species have been previously suggested as the active sites in catalytic ethylene hydrogenation\cite{Li2016sintering,Shabbir2020} and ethylene dimerization.\cite{Ye2017} 

% Further evidence for highly coordinated Ni is observed from the Ni-Ni peaks. As the structures are reduced (\ce{Ni} atoms become uncoordinated), we start to observe the presence of a \ce{Ni{\Compactcdots}Ni} peak around 2.3 {\AA} which has a distance shorter than the bond of converged bulk \ce{Ni} (2.46 {\AA}). Additionally, we observe a lack of \ce{Ni{\Compactcdots}Ni} peak split around the 3.01 {\AA} and 3.27 {\AA} peaks shown by the experimental dPDF.

%The local coordination of the \ce{Ni} atoms vary significantly as a function of the environmental conditions (Figure~\ref{fig:Ni-structure-diagram}). 
%Given the discrepancies in coordination numbers and dPDFs for structures featuring variable \ce{Ni} content, we shift our focus towards structures with a a fixed \ce{Ni} composition of four (Figure~\ref{fig:phase_diagram_Ni_combined} (b)).


To provide further insight into the structure and composition of the Ni catalyst, dPDFs for structures 1, 3, 4, 5, 6, 7, and 8 (purple, orange, blue, red, and lilac lines), i.e., those that ... state rational for including these in the dPDF figure, are compared with the dPDF from the experimental structure (black line) is provided in Figure~\ref{fig:dPDFs_TandP_trans_Ni}. Key peaks in the experimentally observed dPDF are the single peak at 2.02 {\AA}, split peaks at 3.01 {\AA} and 3.27 {\AA}, and split peaks at 3.78 {\AA} and 4.08 {\AA}. These correspond to \ce{Ni-O}, \ce{Ni{\Compactcdots}Ni}, and \ce{Ni{\Compactcdots}Zr} distances, respectively. 

% The presence of five peaks suggests a Ni coordination number of ca. 5. This rules out structures 4-\ce{Ni4(OH)4} (light blue), 5-\ce{Ni3(OH)2} (red), 6-\ce{Ni2(H)2} (lilac), and 7-\ce{Ni4(OH)4.H2O} (yellow), which all have Ni coordination numbers less than 5. Notably, structures with fewer than four Ni cations (i.e., most of the structures on Figure 2b) cannot achieve such high coordination. As removal of \ce{Ni} occurs via the removal of $\ce{OH^{*}}$ ligands, the presence of $\ce{OH^{*}}$ ligands is vital in keeping the structure together. Structures that contain \ce{Ni} atoms featuring high coordination with either $\ce{OH^{*}}$ or $\ce{H2O^{*}}$ ligands show better agreement with the experimental dPDF (Figure~\ref{fig:dPDFs_TandP_trans_Ni}). 





% everything below this idea should now be removed.. is this all necessary?

% Our models comprising four Ni ions contain two types of \ce{Ni} ions: two that connect to the \ce{Zr} nodes and two that do not (e.g., see Figures~\ref{fig:Ni-MOF-model} (c), \ref{fig:Ni-MOF-model} (d), and \ref{fig:Ni-structure-diagram}). 

% tructures 5-\ce{Ni3(OH)2} (red) and 6-\ce{Ni2(H)2} (lilac) represent the two structures on the phase diagram (Figure~\ref{fig:phase_diagram_Ni_combined} (a)) with Nickel contents less than four and several $\ce{OH^{*}}$ ligands removed. The dPDFs shows an underprediction in the \ce{Ni-O} peak, with 5-\ce{Ni3(OH)2} (red) overpredicting the \ce{Ni{\Compactcdots}Ni} peaks and 6-\ce{Ni2(H)2} not including any \ce{Ni{\Compactcdots}Ni} peaks (as described by Figure~\ref{fig:dPDFs_TandP_trans_Ni}).

%Based on the modeling, we suggest that the presence of a direct \ce{Ni-Ni} bond is not seen experimentally and that again we expect a high local coordination involving both \ce{OH} or \ce{H2O} ligands. The structures that meet these criteria show better agreement to the experimental dPDFs with the bulk of these structure observed at lower \ce{H2} partial pressures. However, we do not find a structure that contains perfect agreement with the experimental dPDF.

% This difference between experiments and simulations could be due to kinetics or other things not included in the models.



%Of interest is the catalyst composition and structure under operating conditions, i.e., 0.05~bar \ce{H2} and $T \le$~200~K, signified by the green dot on each phase diagram (Figure~\ref{fig:phase_diagram_Ni_combined}). The structure that minimizes $F^{(3)}$ at these conditions, according to Figure~\ref{fig:phase_diagram_Ni_combined} (a), i.e., where the compositions of hydride, hydroxyl, and water ligands as well as \ce{Ni} atoms are allowed to vary, is 6-\ce{Ni2(H)2} (lilac) (from Figure~\ref{fig:Ni-structure-diagram}). This structure features two isolated single \ce{Ni} ion hydride structures, similar to models used in prior catalytic studies.\cite{Li2016sintering,Shabbir2020,Hackler2020} However, the dPDF for this structure is in stark contrast with the experimental dPDF, given its low \ce{Ni} coordination number of ca. 3. Further, the dPDF for structure 6-\ce{Ni2(H)2} (lilac) does not match well with the experimentally observed one. In fact, given that removal of \ce{Ni} only occurs after significant depletion of $\ce{OH^{*}}$, which in turn decreases the Ni coordination number, the only structures from our library that have the possibility of matching the experimental dPDF are those with four Ni ions. Hence, Figure~\ref{fig:phase_diagram_Ni_combined} (b) provides a phase diagram where the number of \ce{Ni} atoms is held fixed at 4.

%While several structures on this phase diagram exhibit \ce{Ni-O} peaks in good agreement with experiment, only one, 8-\ce{Ni4(OH)5(H).4H2O} (teal in Figure~\ref{fig:Ni-structure-diagram}), exhibits double \ce{Ni{\Compactcdots}Ni} peaks. This ``teal" structure features hydride ($\ce{H^{*}}$), hydroxyl ($\ce{OH^{*}}$), and water ($\ce{H2O^{*}}$) ligands as well as additional \ce{H2O} molecules that form a hydrogen bonded chain between a hydroxyl ligand and a water ligand. \st{and is the thermodynamically preferred structure under conditions of relevance to catalysis}. 

% provides further insight into the structure and composition of the \ce{Ni} catalysts. We compare the dPDFs of all structures present on Figures~\ref{fig:phase_diagram_Ni_combined} (a) and (b) to the dPDF from the experimental structure (black line) as shown in




By definition, $\mu$ are functions of $T$ and $P$. Assuming ideal gas conditions, $\mu_{\text{H}_2(\text{g})}$ is\cite{Reuter2003,Reuter2004,Grundner2015,Paolucci2016,Li2016,Getman2008,Mandal2020}   
\begin{equation}
    \mu_{\text{H}_{2}(\text{g})}(T,P_{\text{H}_{2}}) = E^\text{elec}_{\text{H}_{2}} + E^\text{ZP}_{\text{H}_{2}} + \left[ G_{\text{H}_{2}}(T,P^{o}) - G_{\text{H}_{2}}(0~\text{K},P^{o}) + RT \ln{\left( \frac{P_{\text{H}_{2}}}{P^{o}} \right)} \right] 
    \label{eq:chemicalpotentialrel}
\end{equation}
where $G$ is the Gibbs free energy, $P$ is partial pressure, and $P^{o}$ is the standard pressure (1 bar). We calculate values of $G$ using the NASA Polynomials\cite{Mcbride1993} in the pMuTT\cite{LYM2019106864} Python package. $G(0~\text{K},P^{o})$ is approximated as $G(10~\text{K},P^{o})$ and obtained by extrapolating the NASA Polynomials, following prior work.\cite{Getman2008,Li2016} The \ce{H2O} chemical potential ($\mu_{\text{H}_2\text{O}(\text{g})}$) is computed analogously. Phase diagrams could be constructed in three dimensions, i.e., as functions of $T$, $P_{\text{\ce{H2}}}$, and $P_{\text{\ce{H2O}}}$. However, since \ce{H2O} is not conventionally thought of being present in hydrogenation reaction conditions, in our analysis, we construct two dimensional phase diagrams as functions of $T$ and $P_{\text{\ce{H2}}}$. These phase diagrams are constructed at constant $P_{\text{\ce{H2O}}}$ of $10^{-1}$ bar and \hl{$10^{-1}$ bar}. These values were selected to show how increasing $P_{\text{\ce{H2O}}}$ leads to structures with \ce{Ni} coordination numbers that are commensurate with experimental observations (see Section \hl{XXX}).