%The chemical potential terms, $\mu_{\text{H}^{*}}$ and $\mu_{\text{OH}^{*}}$, include the temperature and pressure dependencies.. We relate $\mu_{\text{H}^{*}}$ and $\mu_{\text{OH}^{*}}$ to $\mu_{\text{\ce{H2}}}^{\text{g}}(T,P)$ and $\mu_{\text{\ce{H2O}}}^{\text{g}}(T,P)$ by assuming each species is in equilibrium with an ideal gas-reservoir. The gas phase chemical potential terms, $\mu_{i}^{g}(T,P)$, are computed by correcting the electronic energy (referenced at $T$=0 K) of an isolated molecule with a chemical potential term, $\Delta \mu_{H_{2}}(T,P)$, that includes all the temperature- and pressure- dependent free-energy contributions (Equation \ref{eq:chemicalpotentialrel} ).

%$\text{i}$ represents either species \ce{H2} or \ce{H2O}. The gas-phase Gibbs free energies ($\Delta G_{\text{i}}(T,P^{o})$) at standard pressure ($P^{o}$) were . For \ce{Ni}, we approximate the chemical potential term ($\mu_{\text{Ni}^{*}}$) using the electronic energy of a bulk \ce{Ni} metal system. 

%We calculate $\mu_{\text{i}}^{\text{g}}(T,P)$ for both \ce{H2} and \ce{H2O} at temperatures ranging from 20 \degree C to 320 \degree C and \ce{H2} partial pressure ranging from $10^{-30}$ bar to $10^{5}$ bar at a fixed \ce{H2O} partial pressure of $10^{-9}$ bar. For each structure in the library, the free energy (Equation \ref{eq:free-energy-trans}) is calculated under range of activation conditions. For a given set of reaction conditions, we determined the most thermodynamically favorable structure by determining which structure minimizes the free energy (Equation \ref{eq:free-energy-trans}).  
%is the untransformed free energy difference between structure $\text{j}$ ($F_{\text{j}}$) and the reference structure ($F_{\text{ref}}$). The $F_{\text{j}}(T,N_{\text{j},\text{H}^{*}},N_{\text{j},\text{OH}^{*}},N_{\text{j},\text{Ni}^{*}})$ term is the free energy of structure $\text{j}$ with configuration of $N_{\text{j},\text{H}^{*}}$, $N_{\text{j},\text{OH}^{*}}$, and $N_{\text{j},\text{Ni}^{*}}$. We approximate the free energy ($F_{\text{j}}$) using the electronic energy from density functional theory (DFT). The $\Delta N$ terms are the difference between $N_{\text{j}}$ and $N_{\text{ref}}$ for each transformed species.
%We investigate how reaction conditions, such as temperature ($T$) and pressure ($P$), influence the structure and composition of a \ce{Ni(II)} supported metal complex on NU-1000. Figure \ref{fig:Ni-MOF-model} shows the NU-1000 MOF and the location of the \ce{Ni(II)} supported metal complex within the small pore of NU-1000 (pore diameter of 8 {\AA}). The proposed model consists of four \ce{Ni(II)} species located within the small-pore as shown in Figure 1 (d) and (e).\cite{PlateroPrats2017,Ye2017} Structurally, the \ce{Ni(II)} atoms are linked together by \ce{\mu3-OH} and \ce{\mu2-OH} ligands that connect adjacent \ce{Zr} nodes of the small-pore. Depending on conditions, the \ce{Ni} atoms may also be coordinated with adsorbed \ce{H2O} species as shown in Figure \ref{fig:Ni-MOF-model} (d) and (e). 

%We select the \ce{Ni4(OH)6.4H2O} model, shown in Figure \ref{fig:Ni-MOF-model} \hl{(d) and (e)}, to be the reference structure for our investigation. While investigating how reaction conditions alter the catalyst structure, all structural and compositional transformations start from the \ce{Ni4(OH)6.4H2O} model, that is we explore the evolution of the composition and structure starting from the \ce{Ni4(OH)6.4H2O} model. Under \ce{H2}, both the \ce{Ni(II)} and \ce{OH} ligands provide sites that can accept dissociated \ce{H} atoms after \ce{H2} adsorption and dissociation. When these sites, defined as \ce{OH^{*}} and \ce{Ni^{*}}, accept dissociated \ce{H} atoms, the composition of the cluster changes to include additional adsorbed \ce{H2O} (\ce{H2O^{*}}) and/or Nickel-hydride (\ce{Ni-H}). However, the adsorbed \ce{H2O} are not fixed to the \ce{Ni} cluster; our modeling includes the subsequent desorption of \ce{H2O} into the gas-phase. All the aforementioned transformations are dependent on the reaction temperature ($T$), and \ce{H2} ($P_{\text{\ce{H2}}}$) and \ce{H2O} ($P_{\text{\ce{H2O}}}$) partial pressures. We model how the cluster structure and composition changes as a function of these different conditions by computing the free energy of compositionally different model structures  relative to a reference structure. We consider the chemical transformed described by the reversible reaction shown below (Equation \ref{eq:chemical-formula}):
%\begin{equation}
%    \begin{split}
%        \ce{Ni4(OH)6.4H2O + xH2 (g) <=> Ni4(OH)_{10-(y+z)}(H)_{4+2x-(x+y)}.yH2O + zH2O (g)} \\
%    \end{split}
%    \label{eq:chemical-formula}
%\end{equation}
%where x defines the \ce{H2} added to the cluster, y is the number of adsorbed \ce{H2O} on the \ce{Ni} cluster, and z is the number of \ce{H2O} in the gas phase. We look at the thermodynamic stability of different modified clusters (\ce{Ni4(OH)_{10-(y+z)}(H)_{4+2x-(x+y)}}) that contain varying compositions of adsorbed (y) and gaseous (z) \ce{H2O} relative to the \ce{Ni4(OH)6.4H2O} model cluster. 
%To determine the stability of the different structural and compositional variations for the \ce{Ni} metal complex, we calculate the relative free energy referenced to the \ce{Ni4(OH)6.4H2O} for each structure in library using \textit{ab initio} thermodynamic analysis. The approach requires a transformation of the free energy from a fixed number of atoms ($N_{\text{i}}$) to a fixed chemical potential ($\mu_{\text{i}}$). The free energy is transformed with respect to $\mu_{\text{H}^{*}}$, $\mu_{\text{OH}^{*}}$, and $\mu_{\text{Ni}^{*}}$ to account for structural and compositional variations in the number of adsorbed \ce{H} species ($N_{\text{H}^{*}}$), \ce{OH}-ligands ($N_{\text{OH}^{*}}$), and \ce{Ni} species ($N_{\text{Ni}^{*}}$). Equation \ref{eq:free-energy-trans} shows the free energy difference ($\Delta F^{(3)}(T,\mu_{\text{H}^{*}},\mu_{\text{OH}^{*}},\mu_{\text{Ni}^{*}})$) between structure $\text{j}$ and the \ce{Ni4(OH)6} reference structure (ref).

%We generate a library of unique, modified structures by systematically adding \ce{H} atoms to the \ce{Ni^{*}} and \ce{OH^{*}} sites on the metal cluster. The \ce{H} addition process creates a new structure that is structurally and compositionally different from the reference structure. The new structure contains either \ce{Ni-H} or \ce{H2O^{*}}. The adsorbed \ce{H2O^{*}} is systematically removed thereby creating new structures. For these new structures, we again add \ce{H} atoms to the \ce{Ni^{*}} and \ce{OH^{*}} sites on the metal cluster and repeat the removal of any adsorbed \ce{H2O}. The iterative process creates a library of structures that are structurally and compositionally unique with \ce{H^{*}}, \ce{OH^{*}}, and \ce{Ni^{*}}. 

% (from a \ce{OH/OH2} pair) thereby giving NU-1000 a net formal charge of $-$2 and the \ce{Ni(II)} catalyst model a net formal charge of $+$2.

% Herein, we call the experimental \ce{Ni(II)} metal complex supported on the NU-1000 as \ce{Ni}-AIM (\underline{A}tomic Layer Deposition \underline{i}n \underline{M}OFs) thereby adopting the same notation as \citeauthor{PlateroPrats2017}\cite{PlateroPrats2017}


% Text clippings after addressing Rachel's comments from February: 

% The resulting crystal structure contains large hexagonal channels (31 {\AA}), triangular channels (10 {\AA}), and small pores ($\sim$10 {\AA}) as shown in Figures~\ref{fig:Ni-MOF-model}a and~\ref{fig:Ni-MOF-model}b.
%To test the influence of temperature and gas phase composition on the \ce{Ni} coordination number, we additionally investigate scenarios where \ce{Ni} are removed from the structure, i.e., by aggregating to form Ni metal. In these circumstances, we take an additional Legendre transformation of $F^{(2)}$ into $F^{(3)}$ by transforming $N_{\ce{Ni}}$ into $\mu_{\ce{Ni}}$, where $\mu_{\text{Ni}} = \mu_{\text{Ni}(\text{s})}$ and $\mu_{\text{Ni}(\text{s})} = E^\text{elec}_\text{Ni(s)}$, following prior work.\cite{Grundner2015} A full description of how $E_\text{Ni(s)}$ is calculated is provided in Supporting Information. Further details about the thermodynamic modeling approach as well as sensitivity analyses for the various modeling decisions are provided in \hl{supporting information}. The Supporting Information also includes phase diagrams simulated at lower pressures of \ce{H2O}.

\st{We find that $E^\text{ZP} + F^\text{vib}$ contribute negligibly to $F$ (see Supporting Information) and hence approximate $F \approx E^\text{DFT}$.}


REMOVED FROM RACHEL'S MARCH EDIT OF THE MANUSCRIPT

% The \ce{Ni4} cluster spans the length of the c-pore of NU-1000 ($\sim$10 {\AA}), and is attached to adjacent nodes of the pore.
