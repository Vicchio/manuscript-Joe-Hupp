% % % % % % % % % % % % % % % % % % % % % % % % % % % % % % % % % % 
% Introduction
% % % % % % % % % % % % % % % % % % % % % % % % % % % % % % % % % % 

\section{Old Introduction}
Catalysts are vital to the present and future of society, with currently 90\% of the world's consumer goods requiring a catalyst during manufacturing.\cite{Hagen2015} Most industrially relevant catalysts are heterogeneous because of their robustness to reaction conditions and ease of recovery. However, standard bulk heterogeneous catalysts suffer from poorly defined active sites that inhibit molecular understandings of the precise catalyst structure and catalytic mechanisms; this lack of knowledge has limited rational catalyst design for bulk metal catalysts. Over the last two decades, a new area of heterogeneous catalysis has emerged involving metal-organic frameworks (MOFs). MOFs, a porous crystalline material formed from inorganic nodes interconnected by organic linkers, function as both a heterogeneous catalysts\cite{Vitillo2019, Hicks2020} and catalyst supports.\cite{Cui2018, Noh2016, Li2017, Song2019, Nguyen2015} MOFs exhibit a diverse platform of tunable structures with well-defined active sites and chemical functionality thus making them an attractive material for single-atom catalysis. 

Within the scientific community, the \ce{Zr} based MOF NU-1000 is commonly used as a catalyst support for 3d metal complexes.\cite{Shabbir2020, Hackler2020, Ortuno2016, Pellizzeri2018} Structurally, the NU-1000 MOF contains \ce{Zr} nodes interconnected by carbon based pyrene linkers to generate both large hexagonal channels (30 {\AA}), triangular channels (10 {\AA}), and small pores (8 {\AA}). The staggered arrangement of protons forming \ce{OH}- and \ce{OH2}-ligand pairs on the node generate anchoring sites for 3d transition metals.\cite{Planas2014} Deposition of 3d transition metal atoms on the MOF framework occurs in the vapor phase via atomic layer deposition (ALD) in MOFs (AIM) \cite{Mondloch2013} and in condensed phase via solvothermal deposition in MOFs (SIM).\cite{Noh2016} The \ce{OH_{x}}-ligands within the small pore (\~8 A) of NU-1000 preferentially bind the metal species within the framework,\cite{Gallington2016, Rimoldi2017} and the synthesis conditions regulate the active species loading to generate complex active site structures.\cite{Kim2015} Elucidating the exact structure of the active site is challenging; characterization of the metal active sites is suppressed by the bulk framework structure, leading to questions about the structure and composition of the synthesized active site.

Combined experimental characterization and computational modeling has provided insights the active site structure for 3d transition metals supported on the standard NU-1000 MOF framework. The primary focus remains understanding the exact composition of the active site structure that are dependent on the number of metal species. A mononuclear (single-site) model was originally thought to be the active site structure for metals supported on NU-1000.\cite{Li2016sintering,AbdelMageed2019,Gallington2016} Mechanistic investigations into ethylene hydrogenation\cite{Shabbir2020} and propyne partial hydrogenation and isomerization\cite{Hackler2020} use a mononuclear model to high-throughout screen different metal species. \citeauthor{Shabbir2020} demonstrates the different proton tolopogies the mononuclear active site can adopt,\cite{Shabbir2020} while \citeauthor{Hackler2020} investigates two different \ce{H2} splitting pathways on a mononuclear active site.\cite{Hackler2020} A potential limitation of these findings, however, is the usage of a mononuclear model. The structure of the active site has been refined revealing that the metal species supported in the c-pore are more likely to be multinuclear (i.e., consisting of multiple metal species). Using difference envelope density (DED),\cite{Li2017} and differential pair distribution function (dPDF) analysis,\cite{PlateroPrats2017} the formation of multinuclear clusters within the c-pore has been established as the active site for NU-1000. A recent study by \citeauthor{Kim2020}  illustrates the multinuclearity of the active site by determining metal loading per node for different metal precursors.\cite{Kim2020} With metal complex deposition occurring only within the c-pore using the ALD process, metal loading greater than 2 metal atoms per node suggest an active site containing multiple metal atoms. Providing further insight to the multinuclear active site structure for \ce{Ni} complexes, \citeauthor{PlateroPrats2017} suggest the formation of tetranuclear heterobimetallic nanowires spanning the length of the c-pore with the cluster attached to two nodes.\citeauthor{PlateroPrats2017} For \ce{Cu} complexes, \citeauthor{Ikuno2017} proposes a similar trinuclear structure that also spans the c-pore of NU-1000. Experimentally, the nature of the active site for metal complexes on NU-1000 is multinuclear rather than mononuclear; however, the precise structure of the multinuclear active site remains unknown. 

Further complicating our understanding of the active site structure is the structural changes induced  by the reaction environment during both the activation step and the reaction steps. The activation environment, and therefore the structural changes, determine the reaction mechanism, such as dehydrogenation, hydrogenation, dimerization, oligomerization. The reaction environment includes both the temperature and gas phase species. \citeauthor{Kim2015} experimentally determined that the metal loading of \ce{In} deceased from  6 \ce{In} per node to 2 \ce{In} per node as the temperature increased from \SI{80}{Celsius} to \SI{200}{Celsius}.\cite{Kim2015} For ethylene hydrogenation, \citeauthor{Li2016sintering} proposes a mechanism on a mononuclear (single-site) \ce{Ni(II)} complex that is activated with \ce{H2} gas to form a \ce{Ni} hydride by removing a coordinated \ce{OH}-ligand.\cite{Li2016sintering} In a similar \ce{H2} environment, \citeauthor{Halder2020} experimentally demonstrated that above \SI{200}{Celsius} the trinuclear \ce{Cu(II)}-clusters supported on NU-1000 are reduced to zero-valent \ce{Cu(0)} species and stripped away from the node with the NU-1000 framework reverting back to the original unit cell dimensions.\cite{Halder2020} Conversely, when exposed to similar reducing conditions the tetranuclear \ce{Ni(II)} cluster shows only subtle structural variations up to 200 oC\cite{PlateroPrats2017} suggesting the cluster is stable under these conditions and that the initial activation only produces subtle changes into the structure of the cluster. Proposed dimerization mechanisms on the tetranuclear \ce{Ni(II)} cluster activate with \ce{Et2AlCl} to remove \ce{OH}-ligands within the cluster as coordinated \ce{H2O} species and generate the active \ce{C2H5}$^*$ species coordinated to the \ce{Ni} atoms.\cite{Ye2017} Using a mononuclar site \ce{Co} cluster, \citeauthor{Li2017} proposes an activation step involving \ce{O2} gas that removes a coordinated \ce{H2O} and alters the coordination environment of the \ce{Co} atom. The metal complexes supported on MOFs, here specifically NU-1000, demonstrate structural variations depending on reaction conditions. To further our understanding of these metal complex active site structures and design better catalysts, there is a need to understand the transformation of the active site structure depending on the reaction environment. 

Herein our work addresses questions related to the structural changes of the metal complex active site as a function of both temperature and gas phase conditions. By combining Density Functional Theory (DFT) calculations with gas phase empirical models, we use \textit{ab initio} thermodynamic analysis to calculate the stability of a \ce{Ni4}-cluster under different gas phase conditions (\ce{H2O} and \ce{H2}). Modeling reveals the thermodynamic landscape of the \ce{Ni} cluster under different conditions. We further investigate our model by comparing the local structural information of thermodynamic relevant structures to experimental structures. Our results provide structural information about the types of changes exhibited by the cluster to establish more appropriate molecular models, such as the role of \ce{H2O} within the active site.  


% The old paragraph 3 for the introduction

%Our work investigates how the environmental conditions (temperature and gas phase conditions) influence the structure of SSHCs supported on the NU-1000 metal-organic framework (MOF). MOFs, a porous crystalline material formed from inorganic nodes interconnected by organic linkers,\cite{Furukawa2013,Li1999} are commonly used to support 3d transition metal complexes to form SSHCs.\cite{Cui2018,Noh2016,Li2017,Song2019,Nguyen2015,Hackler2020} Metal complexes, comprised of metal atoms (e.g., \ce{Ni(II)}) and coordinating ligands (e.g., \ce{OH}/\ce{OH2}), generate the active site species. However, elucidating the exact structure of the active site is challenging and further complicated by structural changes induced by the reaction environment.\cite{PlateroPrats2017b,Rimoldi2017} We address these ongoing questions about the structure of a \ce{Ni(II)} metal complex supported on NU-1000 when exposed to \ce{H2} gas. Using \textit{ab initio} thermodynamic analysis and dPDF analysis, the local structural information (dPDFs) from thermodynamically-relevant \ce{Ni4}-cluster models are compared directly to experimental \ce{Ni}-clusters under similar conditions to improve our understanding of the \ce{Ni} coordination environment of the multinuclear \ce{Ni(II)} metal complex. Our results provide structural information about the types of changes exhibited by the cluster, such as the role of \ce{H2O} within the active site, to establish more appropriate molecular models.  

% % % % % % % % % % % % % % % % % % % % % % % % % % % % % % % % % %
% 2021-10-13: Random text from Introduction after Rachel's edits.
%
% This was text that was removed because it's redundant. 
% % % % % % % % % % % % % % % % % % % % % % % % % % % % % % % % % %



% and coordinating ligands (e.g., \ce{OH}/\ce{OH2}), generate the active site structure. 

% from thermodynamically-relevant \ce{Ni4}-cluster models are compared directly to experimental Nickel clusters under similar environmental conditions