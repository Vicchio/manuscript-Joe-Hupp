% % % % % % % % % % % % % % % % % % % % % % % % % % % % % % % % % % 
% Introduction
% % % % % % % % % % % % % % % % % % % % % % % % % % % % % % % % % % 

\section{Old Introduction}
Catalysts are vital to the present and future of society, with currently 90\% of the world's consumer goods requiring a catalyst during manufacturing.\cite{Hagen2015} Most industrially relevant catalysts are heterogeneous because of their robustness to reaction conditions and ease of recovery. However, standard bulk heterogeneous catalysts suffer from poorly defined active sites that inhibit molecular understandings of the precise catalyst structure and catalytic mechanisms; this lack of knowledge has limited rational catalyst design for bulk metal catalysts. Over the last two decades, a new area of heterogeneous catalysis has emerged involving metal-organic frameworks (MOFs). MOFs, a porous crystalline material formed from inorganic nodes interconnected by organic linkers, function as both a heterogeneous catalysts\cite{Vitillo2019, Hicks2020} and catalyst supports.\cite{Cui2018, Noh2016, Li2017, Song2019, Nguyen2015} MOFs exhibit a diverse platform of tunable structures with well-defined active sites and chemical functionality thus making them an attractive material for single-atom catalysis. 

Within the scientific community, the \ce{Zr} based MOF NU-1000 is commonly used as a catalyst support for 3d metal complexes.\cite{Shabbir2020, Hackler2020, Ortuno2016, Pellizzeri2018} Structurally, the NU-1000 MOF contains \ce{Zr} nodes interconnected by carbon based pyrene linkers to generate both large hexagonal channels (30 {\AA}), triangular channels (10 {\AA}), and small pores (8 {\AA}). The staggered arrangement of protons forming \ce{OH}- and \ce{OH2}-ligand pairs on the node generate anchoring sites for 3d transition metals.\cite{Planas2014} Deposition of 3d transition metal atoms on the MOF framework occurs in the vapor phase via atomic layer deposition (ALD) in MOFs (AIM) \cite{Mondloch2013} and in condensed phase via solvothermal deposition in MOFs (SIM).\cite{Noh2016} The \ce{OH_{x}}-ligands within the small pore (\~8 A) of NU-1000 preferentially bind the metal species within the framework,\cite{Gallington2016, Rimoldi2017} and the synthesis conditions regulate the active species loading to generate complex active site structures.\cite{Kim2015} Elucidating the exact structure of the active site is challenging; characterization of the metal active sites is suppressed by the bulk framework structure, leading to questions about the structure and composition of the synthesized active site.

Combined experimental characterization and computational modeling has provided insights the active site structure for 3d transition metals supported on the standard NU-1000 MOF framework. The primary focus remains understanding the exact composition of the active site structure that are dependent on the number of metal species. A mononuclear (single-site) model was originally thought to be the active site structure for metals supported on NU-1000.\cite{Li2016sintering,AbdelMageed2019,Gallington2016} Mechanistic investigations into ethylene hydrogenation\cite{Shabbir2020} and propyne partial hydrogenation and isomerization\cite{Hackler2020} use a mononuclear model to high-throughout screen different metal species. \citeauthor{Shabbir2020} demonstrates the different proton tolopogies the mononuclear active site can adopt,\cite{Shabbir2020} while \citeauthor{Hackler2020} investigates two different \ce{H2} splitting pathways on a mononuclear active site.\cite{Hackler2020} A potential limitation of these findings, however, is the usage of a mononuclear model. The structure of the active site has been refined revealing that the metal species supported in the c-pore are more likely to be multinuclear (i.e., consisting of multiple metal species). Using difference envelope density (DED),\cite{Li2017} and differential pair distribution function (dPDF) analysis,\cite{PlateroPrats2017} the formation of multinuclear clusters within the c-pore has been established as the active site for NU-1000. A recent study by \citeauthor{Kim2020}  illustrates the multinuclearity of the active site by determining metal loading per node for different metal precursors.\cite{Kim2020} With metal complex deposition occurring only within the c-pore using the ALD process, metal loading greater than 2 metal atoms per node suggest an active site containing multiple metal atoms. Providing further insight to the multinuclear active site structure for \ce{Ni} complexes, \citeauthor{PlateroPrats2017} suggest the formation of tetranuclear heterobimetallic nanowires spanning the length of the c-pore with the cluster attached to two nodes.\citeauthor{PlateroPrats2017} For \ce{Cu} complexes, \citeauthor{Ikuno2017} proposes a similar trinuclear structure that also spans the c-pore of NU-1000. Experimentally, the nature of the active site for metal complexes on NU-1000 is multinuclear rather than mononuclear; however, the precise structure of the multinuclear active site remains unknown. 

Further complicating our understanding of the active site structure is the structural changes induced  by the reaction environment during both the activation step and the reaction steps. The activation environment, and therefore the structural changes, determine the reaction mechanism, such as dehydrogenation, hydrogenation, dimerization, oligomerization. The reaction environment includes both the temperature and gas phase species. \citeauthor{Kim2015} experimentally determined that the metal loading of \ce{In} deceased from  6 \ce{In} per node to 2 \ce{In} per node as the temperature increased from \SI{80}{Celsius} to \SI{200}{Celsius}.\cite{Kim2015} For ethylene hydrogenation, \citeauthor{Li2016sintering} proposes a mechanism on a mononuclear (single-site) \ce{Ni(II)} complex that is activated with \ce{H2} gas to form a \ce{Ni} hydride by removing a coordinated \ce{OH}-ligand.\cite{Li2016sintering} In a similar \ce{H2} environment, \citeauthor{Halder2020} experimentally demonstrated that above \SI{200}{Celsius} the trinuclear \ce{Cu(II)}-clusters supported on NU-1000 are reduced to zero-valent \ce{Cu(0)} species and stripped away from the node with the NU-1000 framework reverting back to the original unit cell dimensions.\cite{Halder2020} Conversely, when exposed to similar reducing conditions the tetranuclear \ce{Ni(II)} cluster shows only subtle structural variations up to 200 oC\cite{PlateroPrats2017} suggesting the cluster is stable under these conditions and that the initial activation only produces subtle changes into the structure of the cluster. Proposed dimerization mechanisms on the tetranuclear \ce{Ni(II)} cluster activate with \ce{Et2AlCl} to remove \ce{OH}-ligands within the cluster as coordinated \ce{H2O} species and generate the active \ce{C2H5}$^*$ species coordinated to the \ce{Ni} atoms.\cite{Ye2017} Using a mononuclar site \ce{Co} cluster, \citeauthor{Li2017} proposes an activation step involving \ce{O2} gas that removes a coordinated \ce{H2O} and alters the coordination environment of the \ce{Co} atom. The metal complexes supported on MOFs, here specifically NU-1000, demonstrate structural variations depending on reaction conditions. To further our understanding of these metal complex active site structures and design better catalysts, there is a need to understand the transformation of the active site structure depending on the reaction environment. 

Herein our work addresses questions related to the structural changes of the metal complex active site as a function of both temperature and gas phase conditions. By combining Density Functional Theory (DFT) calculations with gas phase empirical models, we use \textit{ab initio} thermodynamic analysis to calculate the stability of a \ce{Ni4}-cluster under different gas phase conditions (\ce{H2O} and \ce{H2}). Modeling reveals the thermodynamic landscape of the \ce{Ni} cluster under different conditions. We further investigate our model by comparing the local structural information of thermodynamic relevant structures to experimental structures. Our results provide structural information about the types of changes exhibited by the cluster to establish more appropriate molecular models, such as the role of \ce{H2O} within the active site.  


% The old paragraph 3 for the introduction

%Our work investigates how the environmental conditions (temperature and gas phase conditions) influence the structure of SSHCs supported on the NU-1000 metal-organic framework (MOF). MOFs, a porous crystalline material formed from inorganic nodes interconnected by organic linkers,\cite{Furukawa2013,Li1999} are commonly used to support 3d transition metal complexes to form SSHCs.\cite{Cui2018,Noh2016,Li2017,Song2019,Nguyen2015,Hackler2020} Metal complexes, comprised of metal atoms (e.g., \ce{Ni(II)}) and coordinating ligands (e.g., \ce{OH}/\ce{OH2}), generate the active site species. However, elucidating the exact structure of the active site is challenging and further complicated by structural changes induced by the reaction environment.\cite{PlateroPrats2017b,Rimoldi2017} We address these ongoing questions about the structure of a \ce{Ni(II)} metal complex supported on NU-1000 when exposed to \ce{H2} gas. Using \textit{ab initio} thermodynamic analysis and dPDF analysis, the local structural information (dPDFs) from thermodynamically-relevant \ce{Ni4}-cluster models are compared directly to experimental \ce{Ni}-clusters under similar conditions to improve our understanding of the \ce{Ni} coordination environment of the multinuclear \ce{Ni(II)} metal complex. Our results provide structural information about the types of changes exhibited by the cluster, such as the role of \ce{H2O} within the active site, to establish more appropriate molecular models.  

% % % % % % % % % % % % % % % % % % % % % % % % % % % % % % % % % %
% 2021-10-13: Random text from Introduction after Rachel's edits.
%
% This was text that was removed because it's redundant. 
% % % % % % % % % % % % % % % % % % % % % % % % % % % % % % % % % %



% and coordinating ligands (e.g., \ce{OH}/\ce{OH2}), generate the active site structure. 

% from thermodynamically-relevant \ce{Ni4}-cluster models are compared directly to experimental Nickel clusters under similar environmental conditions


%difficulties in determining the precise active site structure poise significant limitations in catalyst design. 

% The structure of the active site on these various supports is defined by the dispersed metal species, coordinating ligands, and steric environment with these factors contributing to the reactivity and selectivity of the catalyst. The 

% Since the local coordination environment has a strong influence on catalytic performance, learning the precise SSHC active site structure, which is already challenging, is exacerbated by the influence of environmental conditions. 

% Combined experimental and computational approaches address SSHC active site characterization questions when exposed to different environmental conditions.

%Traditionally, chemical reaction on heterogeneous catalysts occurs at the surface of a solid catalyst with a small number of metal atoms participating in the reaction. Unlike in homogeneous catalysts, which feature well-defined isolated active sites and high metal utilization, the exact geometric and electronic nature of a heterogeneous catalyst active site is often unknown or challenging to elucidate.

% These newly envisioned heterogeneous catalytic materials are called

% Recent scientific advancements address limitations in heterogeneous catalysts by designing catalysts with that


% , a porous crystalline material formed from inorganic nodes interconnected by organic linkers,\cite{Furukawa2013,Li1999} are commonly used to support 3d transition metal complexes to form SSHCs.




reference for the general trends here: \cite{Miller2011, Inoglu2010, Zuo2016, Ranea2008}





REMOVED FROM RACHEL'S MARCH EDIT OF THE MANUSCRIPT


% The downside to this uniqueness is that it complicates catalyst design, since the patterns of behavior that are common with traditional bulk metal catalysts are not present.

% However, the same trends, such as how the ligand environment evolves under different conditions or whether or not the metal centers will agglomerate, does not currently exist for SSHCs. 

% For example, \citeauthor{DeRita2019} demonstrates that local coordination of isolated \ce{Pt} atoms deposited on \ce{TiO2} nanoparticles evolve to \ce{(PtO2)_{ads}} species under mild reduction (250 \degree C in \ce{H2}) and to \ce{(PtOH)_{ads}} species under harsh reduction (450 \degree C in \ce{H2}). These structural changes, as determined by both in situ atomic-resolution microscopy and spectroscopy-based characterization supported by first-principle calculations, strongly influence the chemical reactivity for \ce{CO} oxidation.

% Systematically addressing how the local coordination environment changes in responses to variations in environmental conditions is essential when determining structure-function relationships for SSHCs regardless of the support.

% Similarly, reduction environmental conditions activate a \ce{Ni^{II}}-oxo, hydroxo clusters supported metal complex in a MOFs by altering local \ce{Ni} coordination by manipulating the \ce{H}, \ce{O}, \ce{OH}, \ce{H2O} ligands present.\cite{Li2016sintering,Ye2017} Conversely, certain reduction environmental conditions (i.e., 200 \degree C in \ce{H2}) drive a Copper(II)-hydroxo SSHC supported in a MOFs to encapsulated \ce{Cu^{0}} nanoparticles\cite{Halder2020,Mian2020} by removal of any \ce{OH} and \ce{H2O} ligands present.

%  Similar ligated \ce{Ni} catalysts are now found in heterogeneous systems, such as zeolites\cite{2014Finiels} and MOFs.\cite{Ye2017, Bernales2016}

% the \ce{Ni} metal complex ligand environment is sensitive to environmental conditions ($T$, $P_{\text{\ce{H2}}}$, and $P_{\text{\ce{H2O}}}$), and that


%Designing SSHCs requires developing structure-function relationships between catalyst composition and structure with performance. The structures of SSHCs are strongly influenced by operating conditions (e.g., temperature and pressure).\cite{Li2018, DeRita2019} For example, operating conditions modulate the coordinate ligand and structural environments of metal atom centers,\cite{Redfern2018, Kim2015} such as the change in \ce{Pt-O} coordination of single \ce{Pt} atoms supported on \ce{CeO2} (100) due to surface oxygen mobility as a function of temperature.\cite{Daelman2019} Predicting how operating conditions alter any SSHC is non-trivial due to the uniqueness of catalyst, and depends on both the metal and the support. Given catalyst structure drives catalyst function,\cite{DeRita2019} designing SSHCs requires an understanding about the catalyst structure and composition under operating conditions.\cite{Tang2019} This remains an ongoing challenge, with rational catalyst design inhibited by difficulties in determining the precise active site.

%In this work, we seek to learn the composition and structure of a single-site \ce{Ni} catalyst under conditions relevant to catalytic hydrogenation, because of their importance in a variety of reactions, including the Shell higher oligomers process (SHOP)\cite{1988Reuben} and the steam reforming of methane using the Catalytic Rich Gas (CRG) catalyst.\cite{Ross1973} Specifically, we investigate a \ce{Ni} cluster supported on the MOF NU-1000, herein denoted Ni-NU-1000. MOFs are porous crystals comprised of metal oxide/hydroxide nodes connected by organic ``linker" molecules.\cite{Li1999} NU-1000 (Figure~\ref{fig:Ni-MOF-model}a) is comprised of \ce{Zr6(\mu_{3}-O)4(\mu_{3}-OH)4(H2O)4(OH)4} nodes connected with tetratopic 1,3,6,8-tetrakis (p-benzoate) pyrene (TBAPy) linkers. The \ce{Ni} ions are known to anchor to the NU-1000 nodes via the accessible \ce{-OH}/\ce{-OH2} ligands on the \ce{Zr6} nodes, with Atomic Layer Deposition in MOFs (AIM) depositing the \ce{Ni} atoms.\cite{Mondloch2013} Initially, the structure was thought to consist of a single divalent \ce{Ni} ion coordinated zero to three \ce{H2O} ligands, depending on the operating conditions.\cite{Li2016sintering} However, difference envelope density (DED) analysis of powder diffraction data and electron microscopy later revealed that the catalyst is actually comprised of four \ce{Ni} ions per node.\cite{PlateroPrats2017} Computational studies by \citeauthor{Ye2017} then suggested that the catalyst is a four \ce{Ni} ion cluster containing hydroxo (\ce{OH}) ligands attached to adjacent nodes within NU-1000.\cite{Ye2017} For Ni-NU-1000, Ni hydrides (\ce{Ni-H}) have always been considered a key intermediate for both ethylene hydrogenation\cite{Shabbir2020, Li2016sintering} and ethylene dimerization.\cite{Ye2017} For example, in the work by \citeauthor{Ye2017}, the catalyst was assumed to lose all coordinated \ce{H2O} ligands with an \ce{OH}-ligand being replaced by an ethyl (Et) ligand when activated with \ce{Et2AlCl} (as seen in the \hl{Supporting Information}).\cite{Ye2017} The reaction mechanism then proceeds to contain a \ce{Ni-H}. However, the precise ligand environment remains to be resolved. Addressing questions related to the active site structure are important in improving these Ni-NU-1000 catalysts, and the results will be useful for understanding ligand environments in other 3d transition metal ion catalysts as well. 

% For Ni-NU-1000, \ce{Ni} hydrides (\ce{Ni-H}) have always been considered a key intermediate for both ethylene hydrogenation\cite{Shabbir2020, Li2016sintering} and ethylene dimerization.\cite{Ye2017} For example, in the work by \citeauthor{Ye2017}, the catalyst was assumed to lose all coordinated \ce{H2O} ligands with an \ce{OH}-ligand being replaced by an ethyl (Et) ligand when activated with \ce{Et2AlCl} (as seen in the \hl{Supporting Information}).\cite{Ye2017} The reaction mechanism then proceeds to contain a \ce{Ni-H}. 


%In this work, we seek to learn the composition and structure of a single-site \ce{Ni} catalyst under conditions relevant to catalytic hydrogenation, because of their importance in a variety of reactions, including the Shell higher oligomers process (SHOP)\cite{1988Reuben} and the steam reforming of methane using the Catalytic Rich Gas (CRG) catalyst.\cite{Ross1973} Specifically, we investigate a \ce{Ni} cluster supported on the MOF NU-1000, herein denoted Ni-NU-1000. MOFs are porous crystals comprised of metal oxide/hydroxide nodes connected by organic ``linker" molecules.\cite{Li1999} NU-1000 (Figure~\ref{fig:Ni-MOF-model}a) is comprised of \ce{Zr6(\mu_{3}-O)4(\mu_{3}-OH)4(H2O)4(OH)4} nodes connected with tetratopic 1,3,6,8-tetrakis (p-benzoate) pyrene (TBAPy) linkers. The \ce{Ni} ions are known to anchor to the NU-1000 nodes via the accessible \ce{-OH}/\ce{-OH2} ligands on the \ce{Zr6} nodes, with Atomic Layer Deposition in MOFs (AIM) depositing the \ce{Ni} atoms.\cite{Mondloch2013} Initially, the structure was thought to consist of a single divalent \ce{Ni} ion coordinated zero to three \ce{H2O} ligands, depending on the operating conditions.\cite{Li2016sintering} However, difference envelope density (DED) analysis of powder diffraction data and electron microscopy later revealed that the catalyst is actually comprised of four \ce{Ni} ions per node.\cite{PlateroPrats2017} Computational studies by \citeauthor{Ye2017} then suggested that the catalyst is a four \ce{Ni} ion cluster containing hydroxo (\ce{OH}) ligands attached to adjacent nodes within NU-1000.\cite{Ye2017} For Ni-NU-1000, Ni hydrides (\ce{Ni-H}) have always been considered a key intermediate for both ethylene hydrogenation\cite{Shabbir2020, Li2016sintering} and ethylene dimerization.\cite{Ye2017} For example, in the work by \citeauthor{Ye2017}, the catalyst was assumed to lose all coordinated \ce{H2O} ligands with an \ce{OH}-ligand being replaced by an ethyl (Et) ligand when activated with \ce{Et2AlCl} (as seen in the \hl{Supporting Information}).\cite{Ye2017} The reaction mechanism then proceeds to contain a \ce{Ni-H}. However, the precise ligand environment remains to be resolved. Addressing questions related to the active site structure are important in improving these Ni-NU-1000 catalysts, and the results will be useful for understanding ligand environments in other 3d transition metal ion catalysts as well. 


% Here is some of Rachel's feedback about the active site structure here for these catlaysts:

% cataly hydrogenation and dimerization.. be concise.. enough information. 

% elaborate more intentionally about the Ni-H structure...metal hydride active site is present. 
% Further complexities exist when considering differences between the synthesized and activated samples, which the formation of the key \ce{Ni} hydride (\ce{Ni-H}) intermediate for catalytic ethylene hydrogenation\cite{Shabbir2020, Li2016sintering} and dimerization.\cite{Ye2017} 

% There is evidence that suggests the hydrogenation and dimerization.. be concise.. enough information. 

% Think about it as someone not being in ICDC who hasn't read the papers, but is an expert in SSHCs. 

% Hydride is suggests as being the active site for these reactions. Can also reference Ben's paper on the formation of the hydride being relevant in these reactions. 


%may not be a thermodynamic minimum structure in the catalysis, although could be a transient, as hypothesized by \citeauthor{Li2016sintering}.  



\\ \hl{--------------} \\
 Heterogeneous catalysts exhibit dynamic changes in composition and structure as a function of operating conditions that have a profound effect on catalytic performance. For traditional bulk metal heterogeneous catalysts, general trends between composition/structure and function are well established. However, single-site heterogeneous catalysts (SSHCs) where the active site consists of small metal clusters anchored to a solid support the same trends remain undefined. One such SSHC involves a 3d transition metal complex (i.e., \ce{Ni_4O_xH_y}) supported on a porous metal-organic framework (MOF). Herein, we investigate the ligand environment of a \ce{Ni4} cluster supported on the NU-1000 MOF when exposed to reducing conditions (\ce{H2} gas), with the ligand environment varying from \ce{O}, \ce{OH}, \ce{H2O}, and \ce{H} ligands coordinated to the \ce{Ni} atoms. Differential pair distribution functions (d-PDF) analysis and \textit{ab initio} thermodynamic analysis probe potential ligand environments of the \ce{Ni_4O_xH_y} cluster. We generate over 300 different structures featuring different \ce{Ni4} clusters (i.e., different ligand environments) while also accounting for the spin states of the \ce{Ni} atoms. Phase diagrams show the relative location of thermodynamically relevant conditions as a function of \ce{H2} and \ce{H2O} chemical potential terms, and d-PDF analysis enables comparisons in local structural information to experimental data. Structures with \ce{Ni-O} coordination numbers closer to experimental data occur at high \ce{H2O} chemical potentials, suggesting a large \ce{H2O} chemical potential is present within the MOF. Reasonable agreement is seen in the \ce{Ni-O} peak between model and experimental d-PDFs; however, model structures fail to recreate the split \ce{Ni{\Compactcdots}Ni} peaks. Furthermore, d-PDF analysis suggests that asymmetric ligand environments in structures containing high \ce{Ni-O} coordination demonstrate features that are most closely related to the experimental structure, such as the \ce{Ni-O} peak and the split \ce{Ni{\Compactcdots}Ni} peaks. The combined modeling approach demonstrates how the SSHC active site structure is sensitive to the environment conditions, and provide insights into the correct coordination environment of the \ce{Ni4} cluster supported on the NU-1000 MOF thereby establish general trends in the \ce{Ni} ligand environment for these heterogeneous catalytic systems.
