%An important consideration for the experimental structure is that it could represent a distribution of \ce{Ni} metal complex structures. Therefore, the structure that is catalytically isn't the dominate structure. Our work only provides a snapshot of individual structures whereas the experimental dPDF is comprised of all \ce{Ni} structures within the MOF framework. Structures that are similar in energy could be structurally and compositionally different. Experimentally, the dPDF analysis would be unable to differentiate between two structures. Instead the local structural information for both would be combined. Our inability to model a distribution of structures might explain the inability of our modeling to capture the \ce{Ni{\Compactcdots}Ni} and \ce{Ni{\Compactcdots}Zr} peaking splitting. Atomic configurations very similar in energy could be structurally different enough to generate the observed peak splitting. 





% Although 11$^{\blacklozenge}$-\ce{Ni4(OH)4(O).3H2O} (green) doesn't appear on the phase diagram (Figure~\ref{fig:phase_diagram_Ni_combined}), our analysis still suggests that a diverse ligand coordination is present as a result of environmental conditions.

%Within our modeling efforts, we also performed dPDF analysis on structures that were not local minima and thus were not considered in the phase diagrams (Figure~\ref{fig:phase_diagram_Ni_combined}). Structure that aren't local minima (lowest energy) at their specific compositions were not considered during \textit{ab initio} thermodynamic analysis. Only structures that are local minima at their specific compositions could appear on the phase diagram (Figure~\ref{fig:phase_diagram_Ni_combined}). However, we searched the library of structures to identify compositions that demonstrate features resembling the \ce{Ni-O} and \ce{Ni{\Compactcdots}Ni} peaks. One such structure is 11$^{\blacklozenge}$-\ce{Ni4(OH)4(O).3H2O} (green), as seen in Figure~\ref{fig:Ni-structure-diagram} and \hl{Figure SXXX}.  

%Third, our modeling efforts directly demonstrate why consideration of environmental conditions on SSHCs are vital. Figure~\ref{fig:phase_diagram_Ni_combined} provides direct evidence at the structurally diversity of the ligand coordination environment as a function of the environmental conditions for a Nickel metal complex supported on a MOF while considering environmental conditions. 

% Therefore, any structures exhibiting a \ce{Ni{\Compactcdots}Zr} split peak within the 3.78 {\AA} and 4.08 {\AA} range are encouraging; however, we do not evaluate structural performance based on a structures ability to demonstrate \ce{Ni{\Compactcdots}Zr} split peaks.
