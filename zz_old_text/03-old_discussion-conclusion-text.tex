%An important consideration for the experimental structure is that it could represent a distribution of \ce{Ni} metal complex structures. Therefore, the structure that is catalytically isn't the dominate structure. Our work only provides a snapshot of individual structures whereas the experimental dPDF is comprised of all \ce{Ni} structures within the MOF framework. Structures that are similar in energy could be structurally and compositionally different. Experimentally, the dPDF analysis would be unable to differentiate between two structures. Instead the local structural information for both would be combined. Our inability to model a distribution of structures might explain the inability of our modeling to capture the \ce{Ni{\Compactcdots}Ni} and \ce{Ni{\Compactcdots}Zr} peaking splitting. Atomic configurations very similar in energy could be structurally different enough to generate the observed peak splitting. 