%An important consideration for the experimental structure is that it could represent a distribution of \ce{Ni} metal complex structures. Therefore, the structure that is catalytically isn't the dominate structure. Our work only provides a snapshot of individual structures whereas the experimental dPDF is comprised of all \ce{Ni} structures within the MOF framework. Structures that are similar in energy could be structurally and compositionally different. Experimentally, the dPDF analysis would be unable to differentiate between two structures. Instead the local structural information for both would be combined. Our inability to model a distribution of structures might explain the inability of our modeling to capture the \ce{Ni{\Compactcdots}Ni} and \ce{Ni{\Compactcdots}Zr} peaking splitting. Atomic configurations very similar in energy could be structurally different enough to generate the observed peak splitting. 





% Although 11$^{\blacklozenge}$-\ce{Ni4(OH)4(O).3H2O} (green) doesn't appear on the phase diagram (Figure~\ref{fig:phase_diagram_Ni_combined}), our analysis still suggests that a diverse ligand coordination is present as a result of environmental conditions.

%Within our modeling efforts, we also performed dPDF analysis on structures that were not local minima and thus were not considered in the phase diagrams (Figure~\ref{fig:phase_diagram_Ni_combined}). Structure that aren't local minima (lowest energy) at their specific compositions were not considered during \textit{ab initio} thermodynamic analysis. Only structures that are local minima at their specific compositions could appear on the phase diagram (Figure~\ref{fig:phase_diagram_Ni_combined}). However, we searched the library of structures to identify compositions that demonstrate features resembling the \ce{Ni-O} and \ce{Ni{\Compactcdots}Ni} peaks. One such structure is 11$^{\blacklozenge}$-\ce{Ni4(OH)4(O).3H2O} (green), as seen in Figure~\ref{fig:Ni-structure-diagram} and \hl{Figure SXXX}.  

%Third, our modeling efforts directly demonstrate why consideration of environmental conditions on SSHCs are vital. Figure~\ref{fig:phase_diagram_Ni_combined} provides direct evidence at the structurally diversity of the ligand coordination environment as a function of the environmental conditions for a Nickel metal complex supported on a MOF while considering environmental conditions. 

% Therefore, any structures exhibiting a \ce{Ni{\Compactcdots}Zr} split peak within the 3.78 {\AA} and 4.08 {\AA} range are encouraging; however, we do not evaluate structural performance based on a structures ability to demonstrate \ce{Ni{\Compactcdots}Zr} split peaks.




% below is everything that Rachel has written.. I'm just parsing the information down
Of the structures represented in Figures 2a and 2b, only 7 and 8 exhibit peak splitting observed experimentally. While these structures capture double peaks for the \ce{Ni{\Compactcdots}Ni} (8) or \ce{Ni{\Compactcdots}Zr} (7) distances, they are still not a good match with experiment, as they comprise Ni with coordination number smaller than 5 (7) or broaden the Ni-O peak (8) and underpredict the \ce{Ni-O} distance (7 and 8). Further, these structures contains a single \ce{Ni{\Compactcdots}Ni} (7) or \ce{Ni{\Compactcdots}Zr} (8) peak. Hence, to provide further insight into the ligand composition of the observed structure, we searched our structural database for structures that more closely matched the experimental dPDF. The structure from our library that gave the best agreement with the experimental dPDFs is 11$^{\blacklozenge}$-\ce{Ni4(OH)4(O).3H2O} (green), shown in Figures 3 and 4. It should be noted that this structure is not a thermodynamic minimum, according to our analysis, which is why it does not appear on the phase diagram. This structure comprises hydride, hydroxyl, and water ligands and additionally includes a \ce{\mu_{2}-O} ligand instead of a \ce{\mu_{2}-OH} ligand. The composition of structure 11$^{\blacklozenge}$-\ce{Ni4(OH)4(O).3H2O} (green) is both compositional diverse and asymmetric with ligand coordination. The structure contains $\ce{OH^{*}}$, $\ce{H2O^{*}}$, and even a $\ce{O^{*}}$ ligand. The \ce{Ni} coordination is such that the $\ce{OH^{*}}$ and $\ce{O^{*}}$ link adjacent two \ce{Ni} atoms. Additionally, an $\ce{H2O^{*}}$ ligand occupies the space of a $\ce{OH^{*}}$ ligand (i.e., the $\ce{H2O^{*}}$ ligand is coordinated to two \ce{Ni} atoms). The structure features high \ce{Ni} coordination, which as previously discussed, is an important characteristic of structures resemble the experimental structure. From Figure~\ref{fig:dPDFs_TandP_trans_Ni}), the dPDF for 11$^{\blacklozenge}$-\ce{Ni4(OH)4(O).3H2O} (green) shows good agreement for the \ce{Ni-O} peak when compared to the experimental dPDF. Additionally, 11$^{\blacklozenge}$-\ce{Ni4(OH)4(O).3H2O} (green) contains split \ce{Ni{\Compactcdots}Ni} peaks that closely matched the experimental dPDF. For the split \ce{Ni{\Compactcdots}Ni} peaks, 11$^{\blacklozenge}$-\ce{Ni4(OH)4(O).3H2O} shows the best agreement compared to the experimental dPDF. Given that the structure dPDF that best agrees with the experimental dPDF contains the compositional diverse ($\ce{OH^{*}}$, $\ce{H2O^{*}}$, and $\ce{O^{*}}$ ligands) and asymmetric ($\ce{H2O^{*}}$ occupying a `defect' site and $\ce{O^{*}}$ ligand coordinating two \ce{Ni} atoms) ligand coordination suggests that considering alternative structures is important. Although not as pronounced, the green structure also exhibits \ce{Ni{\Compactcdots}Zr} peak splitting. While this structure still does not reproduce the experimental dPDF, it provides clues as to the composition and structure of the Ni cluster under catalytic operating conditions. Specifically, this structure is likely to comprise at least four Ni ions with high coordination that involves diverse ligand coordination. 








% ALL OF THE TEXT BELOW CONTAINS THE TEXT THAT ISN'T IMPORTANT HERE.
% THE TEXT BELOW ARE THINGS THAT RACHEL REMOVED FROM THE MAIN TEXT
% AFTER HER LAST MAJOR ITERATION. 


%The only structure exhibiting split \ce{Ni{\Compactcdots}Ni} peaks is structure \ce{Ni4(OH)5(H).4H2O} (blue), which again features the Nickel-hydride (\ce{Ni-H}) and a bridge of hydrogen bonded \ce{H2O}s. However, only one of these peaks aligns with the experimental peak. The other peak is observed at a larger distance.

% From \citeauthor{Ye2017}, the first coordination sphere of the \ce{Ni4} cluster is thought to be comprised of \ce{OH} and \ce{H2O} ligands with \ce{OH} ligands forming links between \ce{Ni} atoms and \ce{H2O} binding on the ``open" sites of the \ce{Ni} ions in the center of the chain.\cite{Ye2017}

% with the \ce{O} ligand coordinating two \ce{Ni} atoms, lacks a prominent \ce{Ni{\Compactcdots}Ni} peak; a broad peak is observed within the experimental regime




%The observed asymmetries include both diversity in the \ce{Ni} coordination environment (\ce{OH}, \ce{H2O}, \ce{O}) as well as the orientation of certain ligands (mainly, \ce{H2O}). 

%Prior work suggests that a \ce{Ni-H} is the active site in hydrogenation catalysis on a single \ce{Ni} atom.\cite{Li2016sintering, Shabbir2020} Upon exposure to \ce{H2} gas, the Nickel SSHC is thought to dissociate molecular \ce{H2} into atomic \ce{H} with a \ce{Ni-H} forming and the additional \ce{H} atom being adsorbed by one of the \ce{OH} ligands. We explored structures using a similar approach and systematically generated the structures to include \ce{H} (thus forming the \ce{Ni-H} species). Numerous structures in the library of structures contained a \ce{Ni-H}; however, only the \ce{Ni4(OH)5(H).4H2O} (blue) appears on the phase diagram. \citeauthor{Li2016sintering} suggests that the \ce{Ni-H} might exist under a transient state according to EXAFS.\cite{Li2016sintering} Our thermodynamic model supports this claim. While this work does not rule out metal hydrides as active sites, it suggests that other ligands, e.g., \ce{OH} or \ce{H2O} ligands, could participate in the active site for catalysis. The proton (\ce{H}) necessary for hydrogenation could come from either the \ce{OH} or \ce{H2O} ligands with exposure to \ce{H2} gas regenerating these ligands. The lack of a \ce{Ni-H} raises questions about the catalytically active group for the \ce{Ni} metal complex catalyst. 





% paragraph on the structure with good coordination from the structure diagrams
%Furthermore, the \ce{Ni-O} coordination number of the activated structure is measured to be $\sim$5 when exposed to \ce{H2} gas. We use the following information as a reference during \textit{ab initio} thermodynamic analysis to identify appropriate \ce{H2} and \ce{H2O} chemical potential values. We neglect \ce{Ni-O} coordination numbers below 3.3 because the lack of \ce{Ni-O} coordination suggests these aren't relevant structures, which is confirmed by their dPDFs (\hl{see Supporting Information}). All structures appearing on the phase diagram (Figure~\ref{fig:phase_diagram_Ni}) are located within the \hl{Supporting Information.} The structures from \textit{ab initio} thermodynamic analysis that minimize the free energy expression (Eq. \ref{eq:free-energy-trans}) and exhibit \ce{Ni-O} coordination numbers above 3.3 are shown in Figure~\ref{fig:Ni-structure-diagram} (a). We order the structures according to the \ce{Ni-O} coordination numbers, determined by inspecting the \ce{Ni-O} bond distances within each structure. The exact \ce{Ni-O} coordination numbers are reported on Figure~\ref{fig:phase_diagram_Ni}.

% paragraph about the structures 
%The structures with \ce{Ni-O} coordination above 3.3, ranked from highest to lowest, are structures \ce{Ni4(OH)6.4H2O} (green), \ce{Ni4(OH)6.3H2O} (purple), \ce{Ni4(OH)6.2H2O} (cyan), \ce{Ni4(OH)5(H).4H2O} (blue), \ce{Ni4(OH)4(O)} (yellow), and \ce{Ni4(OH)4.2H2O} (orange). The specific \ce{Ni} ligand environments are presented in Figure~\ref{fig:Ni-structure-diagram} (a), with structures exhibiting different \ce{Ni-O} coordination environments. The structures exhibit different compositions featuring \ce{OH}, \ce{H2O}, \ce{H}, and \ce{O} ligands coordinated to the \ce{Ni} atoms. Structures \ce{Ni4(OH)6.4H2O} (green), \ce{Ni4(OH)6.3H2O} (purple), and \ce{Ni4(OH)6.2H2O} (cyan) are the same base \ce{Ni4(OH)6} structure, but contain different \ce{H2O} content (Figure~\ref{fig:Ni-structure-diagram} (a)). Structure \ce{Ni4(OH)5(H).4H2O} (blue) is unique in that it contains both a Nickel-hydride (\ce{Ni-H}) and a bridge of hydrogen bonded \ce{H2O}s. Structure \ce{Ni4(OH)4(O)} (yellow) is the only structure featuring an \ce{O} ligand on the phase diagram, and structure \ce{Ni4(OH)4.2H2O} (orange) features a loss of \ce{OH} ligands coordinated to three different \ce{Ni} atoms. The diverse ligand coordination environment for \ce{Ni} is captured by the structures appearing on the phase diagram (Figure~\ref{fig:phase_diagram_Ni}). 



% paragraph talking about the dPDF of the structures on the phase diagram with higher coordination
%Our analysis primarily focuses on the \ce{Ni-O} and \ce{Ni{\Compactcdots}Ni} seen in the experimental dPDF (Figure~\ref{fig:dPDFs-graphic} (a)), where we compare the structures with coordination above 4.0 shown in Figure~\ref{fig:dPDFs-graphic} (b). Selecting high \ce{Ni-O} coordination is valid, given the mismatch in peak characteristics for 
%structures with low \ce{Ni-O} (\hl{see Supporting Information}). We observe large deviations in peak positions for both the \ce{Ni-O} peak and the split \ce{Ni{\Compactcdots}Ni} peaks at low coordination. As expected, we observe much better agreement between model structures and the experimental dPDF with structures that feature higher \ce{Ni} coordination. 

 

% Paragraph talking about the asymmetry in ligand environment
%A closer inspection of the split peaks in the experimental dPDF at 3.01 {\AA} and 3.27 {\AA} suggests asymmetry in the \ce{Ni} coordination environments. Most structures appearing on the phase diagram (Figure~\ref{fig:phase_diagram_Ni}) are symmetric in their ligand coordination (as showed by the structures located in Figure~\ref{fig:Ni-structure-diagram} (b)). The symmetry in the ligand environment results in symmetric interatomic distances between the different \ce{Ni} species, thereby leading to single peaks \ce{Ni{\Compactcdots}Ni} instead of multiple \ce{Ni{\Compactcdots}Ni} in the 3.01 {\AA} and 3.27 {\AA} range. Our library included structure with asymmetric \ce{Ni} coordination (\hl{as shown in the Supporting information}; however, these structures were not thermodynamic mimima under any explored conditions in \textit{ab initio} thermodynamic analysis. 

%\section{Discussion}
%%%%%%%%%%%%%%%%%%%%%%%%%%%%%%%%%%%%%%%%%%%%%%%%%%%%%%%%%%%%%%%%%%%%%
%% Discussion
%%%%%%%%%%%%%%%%%%%%%%%%%%%%%%%%%%%%%%%%%%%%%%%%%%%%%%%%%%%%%%%%%%%%%

%In general, the simulated structures tend to give good agreement with the \ce{Ni-O} peak when compared to the experimental peak when considering appropriate \ce{Ni-O} coordination numbers. The \ce{Ni-O} peak corresponds to the first coordination shell of the \ce{Ni} atoms; as long as \ce{OH} and \ce{H2O} ligands are present we observe reasonable agreement between the model and experimental \ce{Ni-O} peak. For the \ce{Ni{\Compactcdots}Ni} split peaks, the local coordination of the \ce{Ni} atoms has more of an influence when comparing the simulated and experimental peaks. The \ce{Ni{\Compactcdots}Ni} peaks are from the second coordination shell of the \ce{Ni} atoms. The \ce{Ni{\Compactcdots}Ni} interatomic distances are more sensitive to the ligand environment of the \ce{Ni} atoms compared to the \ce{Ni-O} interatomic distances. None of the structures from \textit{ab initio} thermodynamic analysis exactly recreate the \ce{Ni{\Compactcdots}Ni} peak positions, with most structures only featuring a single 
%\ce{Ni{\Compactcdots}Ni} peak. Only structure \ce{Ni4(OH)5(H).4H2O} (blue) exhibits the split \ce{Ni{\Compactcdots}Ni} peaks. These findings suggests that our model structures aren't preciously recreating the active site environment seen in the experimental system. 

%We identify alternative structures (e.g., structures that are not thermodynamic mimima under any explored conditions) that show reasonable agreement for both the \ce{Ni-O} peak and \ce{Ni{\Compactcdots}Ni} peaks. The composition of structure \ce{Ni4(OH)4)(O).3H2O} (red) from Figure~\ref{fig:Ni-structure-diagram} (b) is both compositional diverse and contains asymmetric ligand coordination. This structure comprises of \ce{OH} and \ce{H2O} ligands, and additionally includes a \ce{\mu_{2}-O} ligand instead of a \ce{\mu_{2}-OH} ligand. The \ce{Ni} coordination is such that the $\ce{OH}$ and $\ce{O}$  ligands link adjacent two \ce{Ni} atoms. Additionally, an $\ce{H2O}$ ligand occupies the space of a $\ce{OH}$ ligand (i.e., the $\ce{H2O}$ ligand is coordinated to two \ce{Ni} atoms). The alternative structure shows split \ce{Ni{\Compactcdots}Ni} peaks with unequally heights, which differs from what was observed experimentally. 

%Given that the structure dPDF that best agrees with the experimental dPDF contains the compositional diverse (\ce{OH}, \ce{H2O}, and \ce{O} ligands) with asymmetric \ce{H2O} ligands and \ce{O} ligand coordinating two \ce{Ni} atoms) ligand coordination suggests that considering alternative structures is important. Although not as pronounced, this structure also exhibits \ce{Ni{\Compactcdots}Zr} peak splitting. While this structure still does not precisely reproduce the experimental dPDF, \ce{Ni4(OH)4)(O).3H2O} (red) provides clues as to the composition and structure of the \ce{Ni} cluster under catalytic operating conditions. Specifically, this structure is likely to comprise at least four \ce{Ni} atoms with high coordination that involves diverse ligand coordination.



%For example, the experimentally observed \ce{Ni{\Compactcdots}Zr} split peaks could be caused by node distortions (i.e., deformations in the MOF crystalline structure). As presently formulated, our model fails to consider any deformations of the MOF crystalline structure. Our unit cell parameters are fixed throughout the entire simulation. As the unit cell dimensions and shapes are held fixed in our models, our models are better able to capture asymmetry due to different ligand environments, but not due to structural effects of the MOF itself. Possible deviations include different coordinating ligands between the two atoms. We did consider variations to the ligands that coordinate directly to the MOF node (\ce{Zr}) and the Nickel SSHC (\ce{Ni}). However, these structures were not thermodynamic minima under any of the explored conditions.
