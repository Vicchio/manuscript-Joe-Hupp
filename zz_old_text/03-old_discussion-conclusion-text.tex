%An important consideration for the experimental structure is that it could represent a distribution of \ce{Ni} metal complex structures. Therefore, the structure that is catalytically isn't the dominate structure. Our work only provides a snapshot of individual structures whereas the experimental dPDF is comprised of all \ce{Ni} structures within the MOF framework. Structures that are similar in energy could be structurally and compositionally different. Experimentally, the dPDF analysis would be unable to differentiate between two structures. Instead the local structural information for both would be combined. Our inability to model a distribution of structures might explain the inability of our modeling to capture the \ce{Ni{\Compactcdots}Ni} and \ce{Ni{\Compactcdots}Zr} peaking splitting. Atomic configurations very similar in energy could be structurally different enough to generate the observed peak splitting. 





% Although 11$^{\blacklozenge}$-\ce{Ni4(OH)4(O).3H2O} (green) doesn't appear on the phase diagram (Figure~\ref{fig:phase_diagram_Ni_combined}), our analysis still suggests that a diverse ligand coordination is present as a result of environmental conditions.

%Within our modeling efforts, we also performed dPDF analysis on structures that were not local minima and thus were not considered in the phase diagrams (Figure~\ref{fig:phase_diagram_Ni_combined}). Structure that aren't local minima (lowest energy) at their specific compositions were not considered during \textit{ab initio} thermodynamic analysis. Only structures that are local minima at their specific compositions could appear on the phase diagram (Figure~\ref{fig:phase_diagram_Ni_combined}). However, we searched the library of structures to identify compositions that demonstrate features resembling the \ce{Ni-O} and \ce{Ni{\Compactcdots}Ni} peaks. One such structure is 11$^{\blacklozenge}$-\ce{Ni4(OH)4(O).3H2O} (green), as seen in Figure~\ref{fig:Ni-structure-diagram} and \hl{Figure SXXX}.  

%Third, our modeling efforts directly demonstrate why consideration of environmental conditions on SSHCs are vital. Figure~\ref{fig:phase_diagram_Ni_combined} provides direct evidence at the structurally diversity of the ligand coordination environment as a function of the environmental conditions for a Nickel metal complex supported on a MOF while considering environmental conditions. 

% Therefore, any structures exhibiting a \ce{Ni{\Compactcdots}Zr} split peak within the 3.78 {\AA} and 4.08 {\AA} range are encouraging; however, we do not evaluate structural performance based on a structures ability to demonstrate \ce{Ni{\Compactcdots}Zr} split peaks.




% below is everything that Rachel has written.. I'm just parsing the information down
Of the structures represented in Figures 2a and 2b, only 7 and 8 exhibit peak splitting observed experimentally. While these structures capture double peaks for the \ce{Ni{\Compactcdots}Ni} (8) or \ce{Ni{\Compactcdots}Zr} (7) distances, they are still not a good match with experiment, as they comprise Ni with coordination number smaller than 5 (7) or broaden the Ni-O peak (8) and underpredict the \ce{Ni-O} distance (7 and 8). Further, these structures contains a single \ce{Ni{\Compactcdots}Ni} (7) or \ce{Ni{\Compactcdots}Zr} (8) peak. Hence, to provide further insight into the ligand composition of the observed structure, we searched our structural database for structures that more closely matched the experimental dPDF. The structure from our library that gave the best agreement with the experimental dPDFs is 11$^{\blacklozenge}$-\ce{Ni4(OH)4(O).3H2O} (green), shown in Figures 3 and 4. It should be noted that this structure is not a thermodynamic minimum, according to our analysis, which is why it does not appear on the phase diagram. This structure comprises hydride, hydroxyl, and water ligands and additionally includes a \ce{\mu_{2}-O} ligand instead of a \ce{\mu_{2}-OH} ligand. The composition of structure 11$^{\blacklozenge}$-\ce{Ni4(OH)4(O).3H2O} (green) is both compositional diverse and asymmetric with ligand coordination. The structure contains $\ce{OH^{*}}$, $\ce{H2O^{*}}$, and even a $\ce{O^{*}}$ ligand. The \ce{Ni} coordination is such that the $\ce{OH^{*}}$ and $\ce{O^{*}}$ link adjacent two \ce{Ni} atoms. Additionally, an $\ce{H2O^{*}}$ ligand occupies the space of a $\ce{OH^{*}}$ ligand (i.e., the $\ce{H2O^{*}}$ ligand is coordinated to two \ce{Ni} atoms). The structure features high \ce{Ni} coordination, which as previously discussed, is an important characteristic of structures resemble the experimental structure. From Figure~\ref{fig:dPDFs_TandP_trans_Ni}), the dPDF for 11$^{\blacklozenge}$-\ce{Ni4(OH)4(O).3H2O} (green) shows good agreement for the \ce{Ni-O} peak when compared to the experimental dPDF. Additionally, 11$^{\blacklozenge}$-\ce{Ni4(OH)4(O).3H2O} (green) contains split \ce{Ni{\Compactcdots}Ni} peaks that closely matched the experimental dPDF. For the split \ce{Ni{\Compactcdots}Ni} peaks, 11$^{\blacklozenge}$-\ce{Ni4(OH)4(O).3H2O} shows the best agreement compared to the experimental dPDF. Given that the structure dPDF that best agrees with the experimental dPDF contains the compositional diverse ($\ce{OH^{*}}$, $\ce{H2O^{*}}$, and $\ce{O^{*}}$ ligands) and asymmetric ($\ce{H2O^{*}}$ occupying a `defect' site and $\ce{O^{*}}$ ligand coordinating two \ce{Ni} atoms) ligand coordination suggests that considering alternative structures is important. Although not as pronounced, the green structure also exhibits \ce{Ni{\Compactcdots}Zr} peak splitting. While this structure still does not reproduce the experimental dPDF, it provides clues as to the composition and structure of the Ni cluster under catalytic operating conditions. Specifically, this structure is likely to comprise at least four Ni ions with high coordination that involves diverse ligand coordination. 
